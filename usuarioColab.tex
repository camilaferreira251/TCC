Usuário Colab

O novo ambiente do Portal do Software Público disponibilizará um ambiente prático
que será composto por uma lista de e-mail.

No primeiro acesso a este ambiente prático o líder da comunidade deverá criar seu
login fornecendo nome, sobrenome, e-mail e nome de usuário. Após este pequeno 
cadastro já será possível fazer a autenticação no ambiente, no primeiro acesso
será pedida a confirmação do e-mail cadastrado onde o usuário colocará sua senha 
de acesso, feito isso terá sua autenticação autorizada. Este procedimento ocorrerá 
somente no primeiro acesso, após este a autenticação ocorrerá normalmente.

Criar e se vincular a Grupo

Todos os ambientes práticos deverão conter uma lista de discussão que deverá ser criada
no primeiro acesso do usuário líder da comunidade prática. Para a criação desta lista
[TODO: definir os passos para a criação de grupos no  colab]

Tendo criado a lista de e-mail/fórum, também chamada de Grupo no Colab, outros
usuários poderam se associar a este grupo. Para tanto será necessário que o 
usuário tenha feito o login no ambiente e siga os seguintes passos:

\begin{itemize}
\item Clique no botão com o ícone do usuário no canto direitoda tela; 
\item Escolha a opção 'My profile';
\item Clique no botão 'Group Membershio'
\item Escolha um ou mais grupos os quais deseja participar
\item Clique na opção 'Update Subscriptions'
\end{itemize}

Clicando no ícone do canto direito o usuário terá acesso as suas últimas contrbuições
nas listas e no mesmo lugar seu perfil que poderá ser alterado na opção 'Edit Profile'.

Criação de Blog

Para que um usuário crie um blog no ambiente será feita um submissão seguindo os 
seguintes passos:

\begin{itemize}
\item Logar com o seu usuário
\item Clicar em "Blogs" (barra superior)
\item Clicar no botão "Submit a blog" (lado direito, dentro do bloco "Source b
logs"
\item Adicionar informações do novo blog nos campos apresentados (url, nome, 
nome do autor e feed url)
\item Submissão de blog feito com sucesso (aparecerá a seguinte mensagem: 
"Thank you. Your application has been accepted and will be reviewed by admin 
in the near time."
\end{itemize}

Após esta submissão o administrador do ambiente avaliará o blog cadastrado
e o aprovará seguindo os passos:

\begin{itemize}
\item Logar como superuser (admin)
\item Clicar em "Requests" dentro do bloco "Feedzilla"
\item Na linha correspondente ao blog em questão clicar em "Process" na coluna 
"Process" (lado direito da página)
\item Verifique as informações do blog a ser cadastrado
\item Clicar no botão "Save" (lado inferior direito da página)
\item Blog criado
\end{itemize}



 




