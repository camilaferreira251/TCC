\chapter{Metodologia}

Para o presente trabalho foi feito um estudo teórico para descobrirmos as origens
de dois tipos diferentes de processos de desenvolvimento, o desenvolvido pelo 
SISP e o que é utilizado por comunidades de software livre para podermos diferenciá-los
e encontrarmos os postos de intercessão e assim propor um modelo que atenda estes dois 
processos.

O processo do SISP citado pede ser fortemente comparado à obra de Taylor, pois 
tem em comum a preocupação em racionalizar, padronizar e prescrever normas que
devem ser seguidas para se chegar em um resultado de valor.
%
Se observa que os 4 princípios que o Taylor defende são vistos no processo do SISP
da seguinte maneira: 
\begin{itemize}
\item Princípio do Planejamento: no processo do SISP são planejados todos as fases
e do projeto antes mesmo de seu início.
\item Princípio do Preparo: No início do projeto já são escolhidas as pessoas 
ou empresa que irá desenvolver o projeto.
\item Princípio do Controle: O processo cria mecanismos para controlar o desenvolvimento
ao longo de todo o projeto.
\item Princípio da Execução: São distribuídas responsabilidades aos executores do projeto. 
\end{itemize}

Este tipo de processo não dá a devida importância aos aspectos humanos de uma organização
se preocupando mais com os cargos ou funções daqueles que irão executar o processo.

O desenvolvimento de software livre possui características distintas do modelo restrito. Portanto,
a relação com o mercado, o processo de desenvolvimento e o produto a ser oferecido devem ser
abordados de maneiras distintas.\cite{meirelles2013metrics}

Com o surgimento da Internet, a facilidade de comunicação e distribuição de 
softwares possibilitou o surgimento de novas formas de trabalho colaborativo.
Aliando esse avanço na comunicação ao uso da modularidade (isto é, a possibilidade 
de divisão do software em componentes desenvolvíveis independentemente) e de 
integradores automáticos das contribuições individuais, passou a ser possível 
envolver colaboradores extremamente diversos em torno de uma grande tarefa. 
As barreiras de entrada para participação diminuíram (pois cada colaborador podia 
selecionar onde ia trabalhar, e a granularidade — tamanho e complexidade — do módulo 
em que iria contribuir), e a qualidade do esforço coletivo pôde aumentar, dada a 
diversidade dos colaboradores.6 Trata-se do movimento do software livre; 
a construção coletiva de uma ampla gama de softwares de qualidade, em constante 
atualização e evolução.\cite{simon2010rossio}

Sabendo que o processo de desenvolvimento de software livre é colaborativo, este 
está mais relacionada com a teoria humanística onde é deixado de lado a preocupação
com os cargos ou máquinas e é dada a preocupação às pessoas que são quem realmente
importa no desenvolvimento do software.


 
