\chapter{Barreiras para contribuir com software livre}

Neste capítulo daremos início ao assunto de barreiras enfrentadas por desenvolvedores
de software livre com base nas barreiras levantadas pelo professor Igor Steinmacher, 
em sua tese de doutorado pela USP.

O software livre se apresenta atualmente como um modelo já consolidado e viável
em um número crescente de aplicações e ambientes. No entanto, restam incertezas
jurídicas e dificuldades, por exemplo com patentes de software, possíveis problemas de
interoperabilidade com plataformas e protocolos criados e baseados em implementações
não-livres e a necessidade de demonstrar a viabilidade econômica do modelo em mais
casos e contextos. Alguns desses desafios estão vinculados às barreiras criadas pelo
modelo e cultura do software restrito~\cite{kon2012software}.

Além dessas dificuldades técnicas enfrentadas por projetos de software livre, existe 
também a dificuldade de se encontrar e estabelecer novos contribuidores para seus os
projetos. Dessa forma o objetivo de sua pesquisa foi encontrar as barreiras encontradas 
por novos contribuidores para começar a contribuir com projetos de software livre~\cite{Steinmacher:2015:SLR:2729291.2729410}.

A pesquisa não tinha interesse em saber o motivo que levava os novos contribuidores
a escolher um projeto de software livre, mas as tarefas que ele tinha que fazer após
ter escolhido um projeto~\cite{Steinmacher:2015:SLR:2729291.2729410}.

O processo de entrada de um novo contribuidor a um projeto de software livre é complexo
e composto de diferentes estágios, pode-se considerar 4 diferentes forças que influenciam 
esse processo: motivação do novo contribuidor, atratividade do projeto, barreiras de entrada
no projeto e retenção do projeto~\cite{Steinmacher:2014:HLO:2593702.2593704, 2049}.

Para entender as barreiras encontradas por novos contribuidores durante a entradas dos
mesmo nos projetos foi feita uma análise qualitativa feita com diferentes estudantes em
diferentes projetos de software livre que foram consultados através de um questionário~\cite{Steinmacher:2014:HLO:2593702.2593704}.

Os dados dos questionários foram analisados conforme a \textit{Grounded Theory},
que se baseia no conceito da codificação, que é o processo de atribuição de um código
para efeitos de classificação ou identificação, e pode ser dividido em 3 passos:

\begin{itemize}

\item Codificação aberta, onde os conceitos são indentificados e suas propriedades e dimensões
são descobertos através dos dados.

\item Codificação \textit{axial}, onde são feitas as conexões dos dados identidicados e 
agrupados de acordo com suas propriedades para representar categorias.

\item Seleção de código, onde cada categoria é identificada e descrita~\cite{strauss1998basics}.

\end{itemize}

No proceso de codificação surgiram 38 barreiras que foram divididas em 7 categorias com suas 
respectivas barreiras:

\begin{itemize}

\item Tarefas para levantar o ambiente: 

	\begin{itemize}
	\item Tarefas para levantar o ambiente; 
	\item Dependências das plataformas;
	\item Problemas para achar o repositório oficial;
	\item Problemas com bibliotecas.
	\end{itemize}

\item Tarefas de código:
	
	\begin{itemize}
	\item Tamanho do código;
	\item Má qualidade do código;
	\item Falta de padrões de código;
	\item Problemas para entender o código;
	\item Código desatualizado.
	\end{itemize}

\item Problemas com a documentação:

	\begin{itemize}
	\item Falta de documentação;
	\item Falta de documentação da estrutura do projeto;
	\item Falta de documentação para levantar o ambiente;
	\item Falta de documentação do processo de contribuição no projeto;
	\item Documentação desatualizada;
	\item Documentação confusa;
	\item Documentação espalhada;
	\item Falta de comentários no código;
	\item Falta de documentação de \textit{Design};
	\item Falta de documentação do código.
	\end{itemize}

\item Comportamento do novo contribuidor:

	\begin{itemize}
	\item Falta de compromisso do novo contribuidor;
	\item Subestimar o desafio de começar a contribuir;
	\end{itemize}

\item Conhecimentos técnicos do novo contribuidor:

	\begin{itemize}
	\item Falta de conhecimento nas ferramentas utilizadas pelo projeto;
	\item Conhecimento prévio em ferramentas de controle de versão;
	\item Dificuldade de escolher a ferramenta certa;
	\item Falta de conhecimento nas tecnologias utilizadas no projeto;
	\item Linguagem de programação utilizada;
	\item Curva de aprendizado;
	\item Curva de aprendizado das ferramentas do projeto;
	\item Falta de conhecimento geral.
	\end{itemize}

\item Encontrando uma maneira de começar:
	
	\begin{itemize}
	\item Encontrar uma tarefa para começar;
	\item Encontrar a parte certa do código para trabalhar;
	\item Lista de \textit{bugs} desatualizada.
	\end{itemize}

\item Tarefas de interação social:
	
	\begin{itemize}
	\item Econtrar alguém para ajudar;
	\item Demora na resposta dos e-mails;
	\item Respostas rudes as dúvidas;
	\item Uso de termos intimidadores;
	\item Tarefas de comunicação;
	\item Encontrar um mentor.
	\end{itemize}

\end{itemize} 

A análise qualitativa ajudou a encontrar barreiras e a relação entre as barreiras
que influenciam um novo contribuidoe de um projeto de software livre~\cite{Steinmacher:2014:HLO:2593702.2593704}.

\section{O portal para suporte a novos contribuidores: FLOSSCoach}


