\chapter{Metodologia}
\label{sec:metodologia}

Neste capítulo apresentaremos a metodologia de pesquisa adotada no desenvolvimento 
do trabalho, mostrando os passos utilizados para chegarmos aos resultados.


\section{Problema}

O Software Livre possui um mecanismo de produção colaborativo e dinâmico 
e possui uma organização composta por um conjunto de pessoas que usa e desenvolve 
um único software livre, contribuindo para uma base comum de código-fonte e 
conhecimento~\cite{reis2003caracterizacc}.

Este modelo típico do Software livre se diferencia em muitos aspectos com a forma
que o governo brasileiro desenvolve software, onde se estabelece um rígido processo.


\section{Hipótese}

As boas práticas de desenvolvimento a serem utilizadas em um projeto de software 
dependem do contexto, uma equipe que desenvolve software livre utiliza
meios diferentes para gerenciar o próprio projeto, que pode conter especificidades 
que impedem o uso de determinadas práticas impostas pelo governo federal para controle
do desenvolvimento de software. 

Dessa forma, com base no problema proposto, elaboramos a seguinte hipótese:

\begin{itemize}
\item \emph{H1:Em geral, existem mais barreiras para começar a contribuir 
com um projeto de software público do que com um projeto de software livre.}

\end{itemize}

\section{Pesquisa}

Nós reproduzimos a pesquisa elencada na tese de doutorado do professor Igor 
Steinmacher da maneira que foi apresentada no capítulo \ref{barreirasSL} no ambiente
do SPB.

Para a pesquisa, nós utilizamos as 6 comunidades que acreditávamos ser mais ativas
no Portal do Software Público com base nas participações dos membros nas listas de 
discussão das comunidades e importância estratégica dentro do governo, são elas:

\begin{itemize}

\item Noosfero.gov\footnote{https://softwarepublico.gov.br/social/noosferogov}: 
O noosfero é uma plataforma web para contrução de redes sociais. Está dentro do
SPB como um software livre para a utilização do SERPRO\footnote{http://www.serpro.gov.br/}, 

\item CACIC\footnote{https://softwarepublico.gov.br/social/cacic}: Foi o primeiro 
software público do SPB, é um sistema baseado em agentes que é capaz de obter um 
diagnóstico preciso do parque computacional e fornecer informações de diversos tipos 
de dispositivos e softwares presentes em um computador.

\item SAE - Sistema de apoio educacional\footnote{https://softwarepublico.gov.br/social/sae}:
É um projeto antigo dentro do SPB, utilizado em processos educacionais personalizados no 
ensino presencial e a distância.  

\item I-Educar\footnote{https://softwarepublico.gov.br/social/i-educar}: Software público de 
gestão escolar utilizado por diversos estados, incluindo o Distrito Federal. Está presente
no SPB desde a versão antiga do portal.

\item E-SIC\footnote{https://softwarepublico.gov.br/social/e-sic-livre}: É uma solução voltada 
para a gestão de atendimento ao público baseado em perguntas e respostas, e visa 
oferecer aos municípios um serviço de pleno acordo com a Lei de Acesso à Informação (LAI).
Um projeto novo, foi lançado já na versão nova do SPB, não passando assim pelos 
probelmas enfrentados pelas comunidades na versão antiga do SPB.

\item SEI - Sistema Eletrônico de informações\footnote{https://softwarepublico.gov.br/social/sei}:
É uma plataforma que engloba um conjunto de módulos e funcionalidades que promovem 
a eficiência administrativa. Não é classificado como software público mas foi contruído com
recursos públicos e utiliza hoje a infraestrutura do SPB para sua manutenção.

\end{itemize}

Nós enviamos a as estas comunidades a mensagem presente no anexo \ref{anexo a}
com o pedido que participassem da pesquisa mostrando o objetivo da mesma. Nosso objetivo 
era atingir àqueles que contribuem com os projetos. Deixamos o questionário disponível 
para que seja respondido por 2 meses, enviando a mensagem apresentda em duas ocasiões e obtivemos
19 respostas que puderam ser utilizadas na pesquisa sendo assim dividida entre os projetos:


\begin{table}[h]
	\centering
	\label{tab01}
	
	\begin{tabular}{ccc}
		\toprule
		\textbf{Projeto} & \textbf{Quantidade de respostas} \\
		\midrule
		Noosfero.gov & 5 \\
		CACIC & 2 \\
		SAE & 1 \\
		I-Educar & 3 \\
		E-SIC & 3 \\
		SEI & 5 \\
		\bottomrule
	\end{tabular}

	\caption{Quantidade de respostas dos desenvolvedores por projetos.}
\end{table}
  
Nós procuramos uma comunidade na qual os desenvolvedores tivessem desistido
de desenvolver o software, pedimos ajuda aos mantenedores do SPB no MPOG e eles nos 
indicaram o projeto OASIS\footnote{https://softwarepublico.gov.br/social/oasis} que 
está ativo no SPB desde o antigo portal mas que a algum tempo não haviam alterações
na comunidade, e então enviamos um questionáo específico para desenvolvedores
que desistiram de contribuir, mas não obtivemos nenhuma resposta. 

Para completar a pesquisa, enviamos outro questionários para alguns dos gestores dos
projetos escolhidos, que são chamados no SPB de coordenadores das comunidades e 
são membros mais experientes no projeto, para que obtivéssemos também a visão 
deles em relação a entrada de novos contribuidores aos projetos bas obtivemos 
apenas 3 respostas;

\begin{table}[h]
	\centering
	\label{tab01}
	
	\begin{tabular}{ccc}
		\toprule
		\textbf{Projeto} & \textbf{Quantidade de respostas} \\
		\midrule
		CACIC & 1 \\
		I-Educar & 1 \\
		E-SIC & 1 \\
		\bottomrule
	\end{tabular}

	\caption{Quantidade de respostas dos coordenadores por projetos.}
\end{table}

Desta forma, no geral conseguimos 22 respostas para serem analisadas. Para esta
análise utilizamos \textit{Ground Theory}, explicada anteriormente no capítulo
\ref{barreirasSL}.


