\chapter{Metodologia}
\label{sec:metodologia}


\section{Problema}

O Software Livre possui um mecanismo de produção colaborativo e dinâmico 
e possui uma organização composta por um conjunto de pessoas que usa e desenvolve 
um único software livre, contribuindo para uma base comum de código-fonte e 
conhecimento~\cite{reis2003caracterizacc}.

Este modelo típico do Software livre se diferencia em muitos aspectos com a forma
que o governo brasileiro desenvolve software, onde se estabelece um rígido processo.


\section{Hipótese}

As boas práticas de desenvolvimento a serem utilizadas em um projeto de software 
dependem do contexto, uma equipe que desenvolve software livre utiliza
meios diferentes para gerenciar o próprio projeto, que pode conter especificidades 
que impedem o uso de determinadas práticas impostas pelo governo federal para controle
do desenvolvimento de software. 

Dessa forma, com base no problema proposto, elaboramos a seguinte hipótese:

\begin{itemize}
\item \emph{H1:Em geral, existem mais barreiras para começar a contribuir 
com um projeto de software público do que com um projeto de software livre.}

\end{itemize}


