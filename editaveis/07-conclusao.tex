\chapter{Conclusão}
\label{conclusao}

Com este trabalho notamos que apesar de os desenvolvedores dos projetos de software
livre e público concetrarem esforços em captar novos desevolvedores para os 
projetos ainda existem barreiras que dificultam este processo. 

Software público possui características que dificultam ainda mais a entrada de 
novos desenvolvedores, pois muitas das vezes os desenvolvedores não sabem 
dferenciar este tipo de projeto de um projeto de software livre além de que
projetos de software público estão disponíveis em uma plataforma específica,
na qual os desenvolvedores não entendem sua estrutura.

Outro entrave nesse processo de entrada que desanima os desenvolvedores
é a hierarquia dentro dos órgãos os quais projetos de software público são 
mantidos. Muitas vezes é imposto para a comunidade que mantém o projeto
o desenvolvimento de novas tarefas que não são de interesse da comunidade
mas sim de determinado órgão do governo. Além de que prioridades dentro do
governo mudam conforme a política e o que é prioridade hoje, amanhã já não é mais
e a tarefa do desenvolvedor fica esquecida e o trabalho é perdido.

Neste trabalho tivemos a oportunidade também de iniciar o desenvolvimento de um
projeto de software livre a partir do início e vivenciamos na prática todas as barreiras
já levantadas para software livre. As primeiras barreiras infrentadas foram as de
conhecimentos técnicos, escolhemos a tecnologia \textit{Ruby on rails} por ser
bastante difundida entre as comunidades de software livre apesar do pouco conhecimento
que tinhamos a respeito, passada a curva inicial de aprendizado e com a ajuda de 
colegas conseguimos transpor esta primeira barreira e desenvolver a versão inicial 
do FlossCoach.

Passada a primeira etapa, iniciamos a comunidade do projeto FlossCoach com os colegas
desenvolvedores na USP e o professor Igor Steinmancher mas nos deparamos com a 
falta de conhecimento em comunidades de software livre, criamos uma lista de email,
e um canal de comunicação de troca de mensagens rápidas além de criar uma organização
e um repositório próprio para o projeto para iniciarmos as colaborações.
Depois de alguns dias sem movimentação no repositório, conversamos com os colegas
da USP e descobrimos que eles haviam criado outro repositório para que eles contribuissem
e somente depois eles iriam passar as contribuições para o repositório oficial e esta
prática não condiz com as boas práticas de projetos de solftware livre, explicamos 
como deveria funcionar a todos os desenvolvidos, inclusive na UNB para continuarmos
o desenvolvimento.

Ao montar a equipe de desenvolvimento na disciplina do professor Paulo também nos deparamos
com mais uma barreira enfrentada pelos desenvolvedores, o projeto não possuía ainda 
documentação e por isso os desenvolvedores tiveram dificuldade em levantar o ambiente 
do projeto em suas máquinas para começar as contribuições. O início do desenvolvimento
na terceira etapa também não foi fácil na UNB pois os desenvolvedores não possuíam
todo o conhecimento técnico necessário para contribuir.
Apesar de não termos uma equipe tão grande, no início tivemos ainda problemas na
comunicação entre os desenvolvedores, chegando até a ter mais de um desenvolvedor 
trabalhando em uma mesma tarefa.

Passado por todos esses desafios no início do desenvolvimento até a consolidação
do projeto podemos reparar na relevância da pesquisa apresentada e o quanto as
informações que serão passadas aos desenvolvedores através do portal FlossCoach poderão
ajudar aos desenvolvedores e as comunidades de software livre e público.

Os objetivos traçados para este trabalho foram atendidos pois conseguimos aprender
e vivenciar na práticas as barreiras para contribuir com projetos de software livre,
mapeamos através dos questionários as barreiras para contribuir com projetos de 
software público e as comparamos com as barreiras para contribuir com projetos
de software livre para responder a nossa questão de pesquisa, além de desenvolver 
uma nova versão do portal FlossCoach. 

Respondendo a questão de pesquisa levantada neste trabalho, \textit{Existem mais barreiras para 
começar a contribuir com um projeto de software público do que com um projeto de software 
livre?} Descobrimos que sim,
temos ainda mais barreiras para contribuir com projetos de software público a contrubuir
com projetos de software livre devido à falta de conhecimento dos desenvolvedores
dos projetos em software livre e público e a rígida hierarquia nos órgãos do governo
que dificulta ainda mais o processo de entrada nas comunidades.

\section{Limitações e trabalhos futuros}

O desenvolvimento do trabalho teve como limitação o tempo, como tivemos apenas um ano
para o desenvolvimento do trabalho não foi possível fazer entrevistas com desenvolvedores mais
experientes conforme o trabalho de \citeonline{steinmancher2015}, ou aguardar mais 
tempo para que mais desenvolvedores respondessem aos questionários,
dessa forma obtivemos 2 respostas de coordenadores de comunidades do SPB como visto na Tabela
\ref{tab02} e 19 respostas de desenvolvedores de projetos de software públicos como mostrado
na Tabela~\ref{tab01}.

A plataforma do SPB desenvolvida pela UNB provê uma API via a ferramenta Noosfero,
que disponibiliza informações referentes aos projetos cadastrados no SPB para
serem utilizadas em outros sistemas, dessa forma uma sugestão para continuação
desse trabalho é a integração da API do SBP no FlossCoach para que o mesmo 
busque dinamicamente as informações dos projetos de software público e as
publique no FlossCoach.

Como trabalho futuro, também poderá ser criada uma instância do FlossCoach para
atender apenas projetos de software público, inculindo nas informações sobre
as barreiras encontradas neste trabalho para iniciar o desenvolvimento de 
projetos de software público.


