\chapter{Conclusão}
\label{conclusao}

Projetos de software público possuem características que dificultam ainda mais a entrada de 
novos desenvolvedores, pois muitas vezes os desenvolvedores não sabem 
diferenciar este tipo de projeto de um projeto de software livre além de que
projetos de software público estão disponíveis em uma plataforma específica,
na qual os desenvolvedores não entendem sua estrutura.

Outro entrave nesse processo de entrada que desanima os desenvolvedores
é a hierarquia dentro dos órgãos os quais projetos de software público são 
mantidos. Muitas vezes é imposto para a comunidade que mantém o projeto
o desenvolvimento de novas tarefas que não são de interesse da comunidade
mas sim de determinado órgão do governo. Além de que prioridades dentro do
governo mudam conforme a política e o que é prioridade hoje, amanhã já não é mais
e a tarefa do desenvolvedor fica esquecida e o trabalho é perdido.

Neste trabalho tivemos a oportunidade também de iniciar o desenvolvimento de um
projeto de software livre a partir do início e vivenciamos na prática todas as barreiras
já levantadas para software livre. As primeiras barreiras enfrentadas foram as de
conhecimentos técnicos, por escolhermos a tecnologia \textit{Ruby on rails} por ser
bastante difundida entre as comunidades de software livre para desenvolver a versão inicial 
do FlossCoach.

Passada a primeira etapa, iniciamos a comunidade do projeto FlossCoach com os colegas
desenvolvedores bolsistas e o professor Igor Steinmancher mas nos deparamos com a 
falta de conhecimento em comunidades de software livre, criamos uma lista de email,
um canal de comunicação de troca de mensagens rápidas além de criar uma organização
e um repositório próprio para o projeto para continuarmos com as colaborações.
Depois de alguns dias sem movimentação no repositório, conversamos com os colegas
bolsistas e nos deparamos com a falta de conhecimento e experiência de desenvolvimento colaborativo,
em outras palavras eles estavam mais acostumados com o desenvolvimento Catedral e
desenvolvemos baseados no Bazar. Para alinhar a forma de desenvolvimento, explicamos 
como deveria funcionar a todos os envolvidos, inclusive na UNB e continuamos
o desenvolvimento.

Ao montar a equipe de desenvolvimento na disciplina do professor Paulo Meirelles também nos deparamos
com mais uma barreira enfrentada pelos desenvolvedores, o projeto não possuía ainda 
documentação e por isso os desenvolvedores tiveram dificuldade em levantar o ambiente 
do projeto em suas máquinas para começar as contribuições. O início do desenvolvimento
na terceira etapa também não foi fácil na UNB pois os desenvolvedores não possuíam
todo o conhecimento técnico necessário para contribuir.
Apesar de não termos uma equipe tão grande, no início tivemos ainda problemas na
comunicação entre os desenvolvedores, chegando até a ter mais de um desenvolvedor 
trabalhando em uma mesma tarefa.

Passado por todos esses desafios no início do desenvolvimento até a consolidação
do projeto podemos reparar na relevância da pesquisa apresentada e o quanto as
informações que serão passadas aos desenvolvedores através do portal FlossCoach poderão
ajudar aos desenvolvedores e as comunidades de software livre e público.

Os objetivos traçados para este trabalho foram atendidos pois conseguimos 
mapear através dos questionários as barreiras para contribuir com projetos de 
software público e as comparamos com as barreiras para contribuir com projetos
de software livre para responder a nossa questão de pesquisa, além de iniciar o desenvolvimento 
e a formatação da comunidade da nova versão do portal FlossCoach. 

Respondendo a questão de pesquisa levantada nesse trabalho, \textit{Existem mais barreiras para 
começar a contribuir com um projeto de software público do que com um projeto de software 
livre?} Descobrimos que sim,
temos ainda mais barreiras específicas para contribuir com projetos de software público a contrubuir
com projetos de software livre argumentamos ser devido a falta de conhecimento dos desenvolvedores
dos projetos em software livre e público e a rígida hierarquia nos órgãos do governo
que dificulta ainda mais o processo de entrada nas comunidades.

Por uma decisão de escopo do trabalho não replicamos a pesquisa de 
\citeonline{steinmancher2015} na íntegra, julgamos não ser necessário 
fazer entrevistas com desenvolvedores mais experientes, ou aguardar mais 
tempo para que mais desenvolvedores respondessem aos questionários, pois 
acreditamos ter indícios suficientes, dessa forma obtivemos 3 respostas de 
coordenadores de comunidades do SPB como visto na Tabela
\ref{tab02} e 19 respostas de desenvolvedores de projetos de software públicos 
como mostrado na Tabela~\ref{tab01}.

Como trabalho futuro, poderá ser criada uma instância do FlossCoach para
atender apenas projetos de software público, incluindo nas informações sobre os projetos
as barreiras encontradas nesse trabalho para iniciar o desenvolvimento de 
projetos de software público. Nessa nova instância também pode ser utilizada a 
API da ferramenta Noosfero que o SPB provê, que disponibiliza informações referentes aos 
projetos cadastrados no SPB para serem utilizadas em outros sistemas, assim com
a integração da API, o FlossCoach poderá buscar dinamicamente as informações dos 
projetos de software público e as publicar nessa nova instância do FlossCoach.



