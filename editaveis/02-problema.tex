\part{Aspectos Gerais}

\chapter[Problema]{Problema}

O Software Livre possui um mecanismo de produção colaborativo e dinâmico 
e possui uma organização composta por um conjunto de pessoas que usa e desenvolve 
um único software livre, contribuindo para uma base comum de código-fonte e 
conhecimento.\cite{reis2003caracterizacc}

Este modelo típico do Software livre se diferencia em muitos aspectos com a forma
que o governo brasileiro desenvolve software, onde se estabelece um rígido processo.

Apesar disso, por fatores sociais e econômicos o governo brasileiro vem
criando políticas de incentivo a software livre, mas a dificuldade está em como
fazer a governança de dois mecanismos de gestão da produção de software antagônicos. 	


