\begin{resumo}
 
Projetos de software público são baseados nos princípios e características dos projetos de 
sofware livre, tais como nível de segurança proporcionado pelo seu uso, eliminação das mudanças
impostas pelo modelo proprietário, independência tecnológica, possibilidade de
auditabilidade além da gratuidade de suas licenças.
%
Neste trabalho procuramos entender as dificuldades dos desenvolvedores em começar a 
contribuir com projetos de software público abordando o processo de 
entrada do desenvolvedor em uma comunidade.
%
O objetivo deste trabalho foi verificar no contexto do software público as barreiras 
já verificadas na literatura e investigar barreiras especificas. 
%
Para cumprir esses objetivos realizamos uma pesquisa com comunidades de software
público e elaboramos um novo modelo de barreiras para desenvolver projetos de 
software público. 
 \vspace{\onelineskip}
    
 \noindent
 \textbf{Palavras-chaves}: software livre. software público. barreiras. desenvolvimento. comunidades.
\end{resumo}
