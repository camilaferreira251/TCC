\begin{resumo}
 
Softwares livres e públicos já estão consolidados como um tipo de software confiável 
devido ao nível de segurança proporcionado pelo seu uso, eliminação das mudanças
impostas pelo modelo proprietário, independência tecnológica, possibilidade de
auditabilidade além da gratuidade de suas licenças entre outras características.
%
Neste trabalho procuramos entender as dificuldades do governo em utilizar software
livre/público abordando o efeito de suas licenças e restrições de uso e como são 
tratado os direitos autorais desses softwares no Brasil.
%
O objetivo deste trabalho é adequar o modelo de desenvolvimento utilizado pelo
governo brasileiro para que possa ser utilizados softwares livre e públicos 
levando em consideração suas boas práticas de desenvolvimento.
%
Para tanto será realizada uma pesquisa bibliográfica onde estudaremos os software
livre e público bem como as licenças utilizadas por esses softwares e a maneira 
com que são aplicadas. Passaremos também pelo processos de desenvolvimento de software
PSW-SISP e suas origens na teoria geral da administração buscando pontos de intersecção
entre esse processo e o método de desenvolvimento das comunidades de software livre e 
público.  

 \vspace{\onelineskip}
    
 \noindent
 \textbf{Palavras-chaves}: software livre. software público. governança.
\end{resumo}
