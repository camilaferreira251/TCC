\begin{resumo}
 
Softwares livres e públicos já estão consolidados como um tipo de software confiável 
devido ao nível de segurança proporcionado pelo seu uso, eliminação das mudanças
impostas pelo modelo proprietário, independência tecnológica, possibilidade de
auditabilidade além da gratuidade de suas licenças entre outras características.
%
Neste trabalho procuramos entender as dificuldades dos desenvolvedores em começar a 
contribuir com projetos de software livre/público abordando todo o processo de 
entrada do desenvolvedor em uma comunidade.
%
O objetivo deste trabalho é analisar e entender as barreiras para que novos desenvolvedores
iniciem o desenvolvimento em projetos de software livre, mapear as barreiras as barreiras
encontradas por desenvolvedores para iniciar o desenvolvimento em projetos de software
público além de comparar as barreiras encontradas com as já conhecidas. 
%
Para cumprir esses objetivos será realizada uma pesquisa com comunidades de software
público e será elaborado um novo modelo de barreiras para desenvolver projetos de 
software público para a partir daí poder estabelecer a comparação.
 \vspace{\onelineskip}
    
 \noindent
 \textbf{Palavras-chaves}: software livre. software público. barreiras. desenvolvimento. comunidades.
\end{resumo}
