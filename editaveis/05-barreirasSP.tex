\chapter{Barreiras para contribuir com software público}

Neste capítulos serão apresentadas as barreiras encontradas para contribuir com
software livre tomando como base a pesquisa para as barreiras para contribuir com
projetos de software público.

Na primeira fase da pesquisa, nós identificamos os conceitos com base nos dados 
dos questionários respondidos pelos desenvolvedores e pelos gestores das comunidades de
sofware público, chegamos a estes conceitos identificando de resposta em resposta 
do questionário aquilo que era identificado como barreira para desenvolver 
software público e obtivemos 63 conceitos:

\begin{itemize}
\item Falta de conhecimento em software público.
\item Falta de conhecimento em software livre.
\item Não saber a diferença entre software livre e público.
\item Peculiaridades das comunidades de software público.
%
\item Desenvolvedores não escolheram estar desenvolvendo o projeto, lhes foi imposto.
\item Obrigação de manter o software no órgão em que trabalha.
\item Hierarquia do órgão.
\item Deixa de desenvolver ao sair do órgão.
%
\item Não sabia por onde começar.
\item Falta de um guia de como começar.
\item Falta de informação de como contribuir.
\item Falta de um manual de como contribuir com o projeto que indicasse o que fazer.
\item Falta de conhecimento nas práticas de desenvolvimento colaborativo.
%
\item Falta de documentação do projeto.
\item Falta de documentação para levantar ambiente.
\item Documentação confusa.
\item Falta de documentação de código.
\item Falta de manual de implantação do software e casos de sucesso em órgãos.
\item Dificuldade de encontrar a documentação do projeto.
%
\item Dificuldade de encontrar repositório oficial.
\item Informação descentralizada.
\item Falta de informações técnicas sobre o projeto.
\item Falta de informação sobre a real aplicação do projeto.
\item Falta de informação em relação à fase do projeto.
\item Falta de informações gerais sobre o projeto.
\item Apresentar os objetivos do projeto de forma clara.
%
\item Falta de lista de necessidades/tarefas do projeto.
\item Falta de canal para reportar erros e bugs no projeto.
\item Mapeamento do nível de dificuldade das tarefas do projeto.
\item Acesso às tarefas do projeto.
\item Organização das tarefas no repositório.
%
\item Falta de suporte a dificuldades.
\item Falta de mentores/líderes nos projetos para auxiliar nas dúvidas.
\item Mal uso dos fóruns de dúvidas.
\item Falta de um canal de comunicação entre os desenvolvedores.
\item Timidez para entrar em contato com a comunidade.
\item Receio em não ser bem recebido pela comunidade.
\item Falta de interação da comunidade.
\item Políticas de uso do fórum de discussão.
\item Dificuldade em lidar com pessoas de outras áreas de formação.
\item Respeito aos diferentes pontos de vista detro do projeto.
%
\item Dificuldade em encontrar normas de uso do portal SPB.
\item Dificuldade de encontrar as normas de compartilhamento de projetos no portal SPB.
\item Dificuldade de interação com os mantenedores do SPB.
\item Dificudade em entender a estrututura do SPB.
\item Dificuldade em entender as funcionalidades do portal SPB.
%
\item Falta de incentivo para retornar as contribuições feitas no projeto para a comunidade.
\item Dificuldade em enviar a melhoria para a comunidade.
\item Falta de padrões de código claros na comunidade para que as melhorias sejam aceitas.
%
\item Dificuldade com ferramenta de controle de versão.
\item Nível de conhecimento em programação.
\item Falta de conhecimentos técnicos.
%
\item Má qualidade do código.
\item Código confuso.
\item Código com muitos erros.
\item Imaturidade do código.
%
\item Dedicação ao projeto.
\item Motivação social.
\item Motivação apenas financeira para contribuir com o software.
\item Proatividade.
\item Comprometimento com o projeto.
\item Colocar os objetivos do projeto à frente dos seus objetivos pessoais com o projeto.
%

\end{itemize}

Após a identificação dos conceitos, partimos para a segunda fase apresentada pela
\textit{Ground Theory}, que como explicado no capítulo \ref{barreirasSL}, demanda que  
façamos a categorização dos conceitos agrupando os conceitos relacionados.

\begin{figure}[h]
	\centering
	\label{fig:SPbarreiras}
		\includegraphics[keepaspectratio=true,scale=0.3]{figuras/Barreiras_Software_Publico.eps}
	\caption{Barreiras para contribuir com Software público.}
\end{figure}
