\chapter[Software Livre]{Software Livre}


O software livre muitas vezes é considerado um fenômeno recente que veio à tona
rapidamente nos últimos anos. No entanto, desde o início da Computação a maior parte
dos desenvolvedores trabalhava da forma que hoje identificamos com o software livre:
Compartilhando código de forma aberta. ~\cite{kon2012software}

Software expressa uma solução abstrata dos problemas computacionais. O software, em um
sistema computacional, é o componente que contém o conhecimento relacionado aos problemas a
que a computação se aplica. Por isso, o software é algo de interesse geral, uma vez que vários
aspectos relacionados a ele ultrapassam as questões técnicas, como por exemplo:

\begin{itemize}

\item O processo de desenvolvimento do software;

\item Os mecanismos econômicos (gerenciais, competitivos, sociais, cognitivos etc.) que regem esse
desenvolvimento e seu uso;

\item O relacionamento entre desenvolvedores, fornecedores e usuários do software;

\item Os aspectos éticos e legais relacionados ao software.

\end{itemize}

O que define e diferencia o software livre do que podemos denominar de software restrito passa
pelo entendimento desses quatro pontos dentro do que é conhecido como o ecossistemado 
software livre. O princípio básico desse ecossistema é promover a liberdade do 
usuário, sem discriminar quem tem permissão para usar um software e seus limites 
de uso, baseado na colaboração e num processo de desenvolvimento aberto. Software 
livre é aquele que permite aos usuários usá-lo, estudálo, modificá-lo e redistribui-lo, 
em geral, sem restrições para tal e prevenindo que não sejam impostas restrições aos 
futuros usuários. ~\cite{meirelles2013metrics}

Estima-se que hoje há centenas de milhões de usuários de software livre no mundo. Se
considerarmos também usuários indiretos, que usam serviços baseados em software livre,
como Google, Amazon ou eBay, esse número é ainda maior.~\cite{sabino2009licenccas}


\subsection{Software Livre no Governo}

O governo vêm adotando um conjunto de iniciativas para implemetação de software
livre no Brasil como reação aos enormes custos gerados à Administração com licensas
de softwares, o Decreto presidencial 18/00 instituiu o Comitê Executivo do Governo
Eletrônico, no intuito de racionalizar gastos com softwares. Para efetivar esse 
objetivo, instituiu o Comitê Técnico para Implementação de Software Livre, por
meio do Decreto nº 29/03. Dentro desse programa. estuda-se a forma (livre) pela
qual o Estado deveria se posicionar na aquisição e distribbuição de seus softwares.

\subsubsection{Lei de direitos autorais}

Boa parte do software atualmente usado e desenvolvido, seja para computadores 
pessoais ou servidores, seja para uso geral ou específico, é disponibilizado sob
licenças restritivas. Essas licenças, em maior ou menor grau, impõem restrições 
ao seu uso, distribuição ou acesso ao código-fonte. Esse tipo de licenciamento é 
possível porque o software está sujeito à proteção da lei a respeito dos direitos 
de autor, que garante ao criador o direito exclusivo de exploração de sua obra. 
Isso lhe permite autorizar ou não determinadas formas de uso do software por parte 
dos usuários. %TODO: citar autores, não possui publicação, artigo paulo com maranhão

A legislação brasileira vê o software menos como produto e mais como expressão 
intelectual, prevendo que os direitos de autor são o mecanismo próprio de proteção 
ao software e excluindo explicitamente patentes como opção (Lei 9609/98 e Lei 9279/98, art. 10).

A legislação autoral e de Propriedade Intelectual reconhece o software como objeto 
de direito do autor, com regime específico dado pela lei do sofftware (Lei nº 9609/98)
subsidiada naquilo que for omissa pela Lei do Direito Autoral (Lei nº 9610/98).
%
Assim, cabe ao titutlar do direito autoral a definição da forma como disporá desse
direito, se em regime prorprietário, ou em regime livre. Essa decisão, certamente,
situa-se fora do âmbito de atuação do estado. %TODO citar autores, revista, juliano maranhão

\subsection{Licensas de Software}

Programas de software livre, em geral, são de fácil acesso. Porém, a simples obtenção
de um programa não significa que a pessoa pode fazer o que quiser com ele. As licenças
de software livre são documentos através dos quais os detentores dos direitos sobre um
programa de computador autorizam usos de seu trabalho que, de outra forma, estariam
protegidos pelas leis vigentes no local. ~\cite{sabino2009licenccas}

O detentor dos direitos patrimoniais sobre um software, quando decide torná-lo livre,
deve escolher os termos em que seu trabalho será distribuído, ou seja, os direitos que ele
estará transferindo para as outras pessoas e quais as condiçes que serão aplicadas. O
documento que formaliza esse ato é a licença, que normalmente é distribuída junto com
o código fonte.~\cite{sabino2009licenccas}

Software aberto pode ser distribuído em uma grande variedade de licenças.
A mais famosa é a GNU General Public License (GPL), usada pela maior
parte do software desenvolvido pela Free Software Foundation \footnote{https://www.fsf.org/pt-br}. 
Software que usa esta licença é também chamado de “software livre” pois o seu 
objetivo é dar toda liberdade a quem possui uma cópia do software para fazer o que
bem entender com ele. Software livre (ou free software) pode ser distribuído
gratuitamente, vendido, alterado, etc. Alguns outros grupos, como por exemplo 
o que desenvolve ACE e TAO [SC99], usam licenças alternativas com o intuito de 
facilitar a produção de eventual software fechado que inclua componentes de 
software aberto. Além disso, ACE e TAO também podem ser adquiridos através de e
mpresas comerciais que oferecem treinamento e suporte técnico mediante o pagamento
de taxas sem, no entanto, prejudicar a liberdade de uso, modificação e distribuição 
gratuita do software. ~\cite{kon2001software}

As licensas de software são separadas em permissivas, recíprocas totais e recíprocas
parciais.

As licensas permissivas impõe poucas restrições ao uso e compartilhamento do
software, não éfeita nenhuma restrição ao licenciamento de trabalhos derivados, 
estes podendo inclusive serem distribuídos sob uma licença fechada. Licenças permissivas são 
uma ótima opção para projetos cujo objetivo é atingir o maior número de pessoas, 
não importando se na forma de software livre ou de software fechado. As principais 
licensas dessa categoria são: BSD\footnote{www.creativecommons.org/licenses/BSD/legalcode},
MIT/X11\footnote{www.opensource.org/licenses/mit-license.php} e Apache\footnote{www.apache.org/licenses/LICENSE-2.0.html}.
~\cite{sabino2009licenccas}

As licenças recíprocas totais determinam que qualquer trabalho derivado precisa 
ser distribuído sob os mesmos termos da licença original. Isso também é chamado de copyleft,
um termo criado pela Free Software Foundation\footnote{www.gnu.org/copyleft}. A idéia do
copyleft é dar permissão a todos para executar, copiar, modificar e distribuir versões
modificadas do programa, mas impedir que sejam adicionadas restrições a essas versões
redistribuıdas. A licença que deu origem à idéia de copyleft foi a General Public 
License, ou GPL\footnote{www.gnu.org/licenses/gpl.html}, da Free Software Foundation.
As principais licensas dassa categoria são as devivadas da GPL: 
GPL2.0\footnote{http://www.gnu.org/licenses/old-licenses/gpl-1.0.html}, 
GPLv3\footnote{www.gnu.org/licenses/gpl-3.0-standalone.html} e a
AGPL\footnote{www.affero.org/oagpl.html},que foi desenvolvida pela empresa 
Affero\footnote{www.affero.org}.~\cite{sabino2009licenccas}

Licenças recíprocas parciais, também chamadas de copyleft fraco, determinam que
modificações do trabalho coberto por elas devem ser disponibilizadas sob a mesma 
licença. Porém, quando o trabalho é utilizado apenas como um componente de outro 
projeto, esse projeto não precisa estar sob a mesma licença. As principais licenças
dessa categoria são: LGPL\footnote{www.fsf.org/licensing/licenses/lgpl.html} e a
Licença Mozilla.~\cite{sabino2009licenccas}

\subsection{Software Público}

O Portal do Software Público Brasileiro - SPB, inaugurado em 2007, na prática, é um sistema
web que se consolidou como um ambiente de compartilhamento de projetos de software. Oferece um
espaço (comunidade) para cada software. A comunidade é composta por fórum, notícias, chat, arma-
zenamento de arquivos e downloads, wiki, lista de prestadores de serviços, usuários, coordenadores,
entre outros recursos. Teve um crescimento expressivo contando, hoje, com mais de 60 comunidades
de desenvolvimento e mais de 200.000 usuários cadastrados. O SPB abrange também, o 4CMBr que
é o grupo de interesse voltado para soluções de tecnologia para municípios, o 5CQualiBr que é um
grupo que trabalha para evoluir a qualidade do Software Público Brasileiro, o 4CTecBr, um portal
destinado a colaboração no desenvolvimento de Tecnologias Livres, o Mercado Público Virtual que
é um grupo de empresas e pessoas que prestam serviço nos softwares ofertados no Portal e o 
AvaliaSPB que avalia a entrada dos softwares candidatos a software público. O ambiente do SPB não
proporciona a integração com ambientes colaborativos externos, especialmente com redes sociais. A
plataforma escolhida na ocasião da criação foi o framework OpenACS, que continua sendo utilizada
na atual versão.

Inicialmente, o propósito do Portal compartilhar os softwares desenvolvidos
no governo visando diminuir os custos de contratação de software, mas se observou
que ao se disponibilizar os softwares rapidamente se formavam comunidades em torno 
daquele software com diversas pessoas colaborando e compartilhando os resultados
obtidos através do uso daquelas soluções. Desta forma algumas cooperativas de 
desenvolvimento de software e empresas privados demonstraram o interesse em
publicar seus softwares na plataforma que estava sendo criada.

No contexto do Governo Federal Brasileiro para se entender o software livre e como usá-lo de
forma mais apropriada dentro do órgão públicos, em pesquisa realizada pelo Ministério de Ciência,
Tecnologia e Inovação, em 2005, entre usuários, desenvolvedores e empresas, figuravam entre os 
fatores que motivam usuários e desenvolvedores a adotar o Software Livre: redução de custos, maior
flexibilidade para adaptação, maior qualidade, maior autonomia do fornecedor e maior segurança.
Entendendo essa lógica e com a finalidade de adaptar, bem como fornecer uma segurança jurídica
mais robusta a legislação de nosso País, o Governo Federal criou em 2005 o modelo do Software
Público Brasileiro (SPB), que entre os usuários estão ofertantes e demandantes de soluções, 
organizados em comunidades, criadas em torno de cada solução de software. A intensidade de participação
varia desde um observador interessado no software até líderes de comunidades. Essa diversidade é
derivada do modelo de produção do software livre, no qual baseou-se o SPB para sua formação. A
percepção do potencial que representava a participação da sociedade no desenvolvimento do software
e o conceito de bem público foram adaptadas do ponto de vista jurídico, assim levando o Ministério
do Planejamento, Orçamento e Gestão (MP) a formular o conceito de software público. Essa base
jurídico-institucional permitiu a criação de um ambiente virtual (um portal) para a disponibilização
de software como software público. Esse modelo é definido por uma rede que se auto-organiza e cuja
produção se caracteriza pela intensa participação colaborativa entre indivíduos, empresa ou 
prestadores de serviço, universidades e instituições interessadas na evolução de um determinado projeto
de software.

O conceito de software público diferencia-se do de software livre em alguns aspectos, destacando-se
a atribuição de bem público ao software. Embora haja algumas diferenças entre o que é um software
livre e um software público brasileiro, há princípios comuns, como a tendência da descentralização na
tomada de decisões, o intenso compartilhamento de informações e os processos de retroalimentação
decorrentes do uso dos artefatos produzidos. Em outras palavras, todo software público é um software
livre também.

Para se disponibilizar um software no Portal do SOftware Público a equipe do Portal 
oferece um documento chamado Manual do Ofertante 
onde estão contidas todas as informações e procedimentos.

O modelo atual de disponibilização de um software como público é
regido pela Instrução Normativa Nº 01 de 17 de janeiro de 2011. 
Segundo esta IN ao se disponibilizar um software público o mesmo estará se tornando um bem público
que pode ser usado por todos.












