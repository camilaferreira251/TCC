\chapter{Introdução}
%\addcontentsline{toc}{chapter}{Introdução}




Com o surgimento da Internet, a facilidade de comunicação e distribuição de 
softwares possibilitou o surgimento de novas formas de trabalho colaborativo.
Aliando esse avanço na comunicação ao uso da modularidade (isto é, a possibilidade 
de divisão do software em componentes desenvolvíveis independentemente) e de 
integradores automáticos das contribuições individuais, passou a ser possível 
envolver colaboradores extremamente diversos em torno de uma grande tarefa. 
As barreiras de entrada para participação diminuíram (pois cada colaborador podia 
selecionar onde ia trabalhar, e a granularidade — tamanho e complexidade — do módulo 
em que iria contribuir), e a qualidade do esforço coletivo pôde aumentar, dada a 
diversidade dos colaboradores. Trata-se do movimento do software livre; 
a construção coletiva de uma ampla gama de softwares de qualidade, em constante 
atualização e evolução~\cite{simon2010rossio}.

DESCREVER A DIFICULDADE DE SE COMEÇAR A COLABORAR COM PROJETOS DE SOFTWARE LIVRE E PÚBLICO

\section{Objetivos}

O objetivo geral deste trabalho é descobrir quais a barreiras enfrentadas por um novato ao 
começar a contribuir com projetos de software público.

Os objetivos específicos são:

\begin{itemize}
\item Mapear as barreiras enfrentadas pelos desenvolvedores ao começar a contribuir
com um projeto de software público;
\item Comparar as barreiras encontradas com as barreiras já conhecidas para começar 
a contribuir com um projeto de sofware livre;
\end{itemize}

\section{Justificativa}

% Verificar necessidade da sessão separada de jutificativa.


