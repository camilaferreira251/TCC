\chapter{Introdução}
%\addcontentsline{toc}{chapter}{Introdução}

Com o surgimento da Internet, a facilidade de comunicação e distribuição de 
softwares possibilitou o surgimento de novas formas de trabalho colaborativo.
Aliando esse avanço na comunicação ao uso da modularidade (isto é, a possibilidade 
de divisão do software em componentes desenvolvíveis independentemente) e de 
integradores automáticos das contribuições individuais, passou a ser possível 
envolver colaboradores extremamente diversos em torno de uma grande tarefa. 
As barreiras de entrada para participação diminuíram (pois cada colaborador podia 
selecionar onde ia trabalhar, e a granularidade — tamanho e complexidade — do módulo 
em que iria contribuir), e a qualidade do esforço coletivo pôde aumentar, dada a 
diversidade dos colaboradores. Trata-se do movimento do software livre; 
a construção coletiva de uma ampla gama de softwares de qualidade, em constante 
atualização e evolução~\cite{simon2010rossio}.

Embora a evolução da internet tenha ajudado a difundir as práticas e ideais do
software livre não eliminou todas as barreiras para que os desenvolvedores possam
contribuir com projetos de software livre. Ainda hoje, podemos verificar que 
existem dificuldades enfrentadas por esses projetos que apenas a difusão da internet 
não foi capaz de derrubar.

A falta de conhecimento técnico e a dificuldade na desenvoltura social dos desenvolvedores
continuam a ser obstáculos quando se pretende fazer parte de uma comunidade de software
livre. Buscando entender essa dificuldade o professor Igor Steinmancher desenvolveu
sua pesquisa para o doutorado levantando, através de uma presquisa qualitativa,
as barreiras que atravessam os desenvolvedores ao tentar se tornar um colaborador 
de um projeto de software livre.

Com o modelo pronto e as barreiras para contribuir com software livre mapeadas o
professor Igor desenvolveu uma aplicação estática onde ele colocou os projetos 
utilizados em sua pesquisa com as informações necessárias para mitigar as barreiras
mas nesta aplicação desenvolvida não é possível, cadastrar dinamicamente novos projetos.

Seguindo os ideais do software livre, o conceito de software público nasceu com o
objetivo de economizar recursos pelo governo brasileiro, com características que 
diferenciam um pouco do software livre comum, o software público possui um
Portal do Software Público Brasileiro onde todos os projetos com esta característica
estão ali colocados e disponíveis para a população e órgãos do governo. Por serem
projetos de interesse do governo, existe um gestor público em um órgão interessado 
nas modificações e manutenções aplicadas aos projetos. Por este motivo este trabalho 
vêm buscar verificar se as características e imposições aplicadas a este tipo de projeto 
como a licensa de software que a este projeto deve ser aplicada, a obrigatoriedade de
estar no SPB ou a existência da figura do gestor público interessado no projeto podem
criar mais barreiras para um desenvolvedor passe a fazer parte de dessa comunidade.

%\section{Objetivos}

O objetivo geral deste trabalho é descobrir quais a barreiras enfrentadas por um 
desenvolvedor ao começar a contribuir com projetos de software público.

Os objetivos específicos são:

\begin{itemize}
\item Analisar e enterder as barreiras enfrentadas pelos desenvolvedores ao começar
a contribuir com um projeto de software livre de acordo com a pesquisa do professor
Igor Steinmancher;
\item Mapear as barreiras enfrentadas pelos desenvolvedores ao começar a contribuir
com um projeto de software público;
\item Comparar as barreiras encontradas com as barreiras já conhecidas para começar 
a contribuir com um projeto de sofware livre.
\end{itemize}

Para cumprir esses objetivos, no Capítulo~\ref{software_livre}, faremos uma 
contextualização sobre software livre e software público, suas diferenças e a 
legislação a respeito desses tipos de software no Brasil. Seguindo para o Capítulo
~\ref{barreirasSL}, discutiremos como foi desenvolvida a pesquisa para levantar
barreiras para contribuir com software livre bem como iremos entender quais são
essas barreiras. No Capítulo~\ref{metodologia} falaremos da metodologia de pesquisa
utilizada no desenvolvimento deste trabalho com os seus resultados apresentados no
Capítulo~\ref{barreiras_publico}, onde mostraremos as barreiras encontradas pelos
desenvolvedores para contribuir com projetos de software público e a comparação com 
as barreiras já mapeadas para software público. Nós explicaremos a parte prática deste
trabalho em detalhes no Capítulo~\ref{desenvolvimento} e finalmente, no Capítulo~\ref{conclusao}
analisaremos os resultados obtidos neste trabalho.


