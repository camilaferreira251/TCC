\chapter{Introdução}
%\addcontentsline{toc}{chapter}{Introdução}

Para o presente trabalho foi feito um estudo teórico para descobrirmos as origens
de dois tipos diferentes de processos de desenvolvimento, o desenvolvido pelo 
SISP e o que é utilizado por comunidades de software livre para podermos diferenciá-los
e encontrarmos os postos de intercessão e assim propor um modelo que atenda estes dois 
processos.

O processo do SISP citado pede ser fortemente comparado à obra de Taylor, pois 
tem em comum a preocupação em racionalizar, padronizar e prescrever normas que
devem ser seguidas para se chegar em um resultado de valor.
%
Se observa que os 4 princípios que o Taylor defende são vistos no processo do SISP
da seguinte maneira: 
\begin{itemize}
\item Princípio do Planejamento: no processo do SISP são planejados todos as fases
e do projeto antes mesmo de seu início.
\item Princípio do Preparo: No início do projeto já são escolhidas as pessoas 
ou empresa que irá desenvolver o projeto.
\item Princípio do Controle: O processo cria mecanismos para controlar o desenvolvimento
ao longo de todo o projeto.
\item Princípio da Execução: São distribuídas responsabilidades aos executores do projeto. 
\end{itemize}

Este tipo de processo não dá a devida importância aos aspectos humanos de uma organização
se preocupando mais com os cargos ou funções daqueles que irão executar o processo.

O desenvolvimento de software livre possui características distintas do modelo restrito. Portanto,
a relação com o mercado, o processo de desenvolvimento e o produto a ser oferecido devem ser
abordados de maneiras distintas~\cite{meirelles2013metrics}.

Com o surgimento da Internet, a facilidade de comunicação e distribuição de 
softwares possibilitou o surgimento de novas formas de trabalho colaborativo.
Aliando esse avanço na comunicação ao uso da modularidade (isto é, a possibilidade 
de divisão do software em componentes desenvolvíveis independentemente) e de 
integradores automáticos das contribuições individuais, passou a ser possível 
envolver colaboradores extremamente diversos em torno de uma grande tarefa. 
As barreiras de entrada para participação diminuíram (pois cada colaborador podia 
selecionar onde ia trabalhar, e a granularidade — tamanho e complexidade — do módulo 
em que iria contribuir), e a qualidade do esforço coletivo pôde aumentar, dada a 
diversidade dos colaboradores. Trata-se do movimento do software livre; 
a construção coletiva de uma ampla gama de softwares de qualidade, em constante 
atualização e evolução~\cite{simon2010rossio}.

Sabendo que o processo de desenvolvimento de software livre é colaborativo, este 
está mais relacionada com a teoria humanística onde é deixado de lado a preocupação
com os cargos ou máquinas e é dada a preocupação às pessoas que são o que realmente
importa no desenvolvimento do software.

Pode-se notar que existe uma grande divergência entre a maneira com que o SISP trata
o desenvolvimento de software e como a comunidade de software livre o trata, dessa forma vamos
por meio deste trabalho investigar como estes modelos de desenvolvimento podem 
se adequar a fim de o governo também desenvolver e utilizar software livre.

\section{Objetivos}

O objetivo geral deste trabalho é definir boas práticas de desenvolvimento de 
software livre que sejam aderentes ao processo de desenvolvimento do governo.

Os objetivos específicos são:

\begin{itemize}
\item Mapear o processo de desenvolvimento utilizado pelo governo;
\item Especificar problemas do software livre no governo;
\item Propor um modelo de colaboração equilibrando as características 
dos processos de governo com as características empíricas do desenvolvimento 
do Software livre/público.
\end{itemize}

\section{Justificativa}

A Administração Pública, em sentido amplo, pode ser entendida como um conjunto 
de entidades e de órgãos incumbidos de realizar a atividade administrativa visando 
à satisfação das necessidades coletivas e segundo os fins desejados pelo Estado.
\cite{coutinho2012uso}

A constituição federal estabelece princípios que devem ser adotados pela administração pública
no exercício de suas funções, quais sejam: legalidade, impessoalidade, moralidade, publicidade, e
eficiência, portanto “Não há “espaço jurídico vazio” dentro do qual a Administração possa escolher
livremente os fins a perseguir e os meios para alcançá-los” 
%
Com efeito, a escolha de softwares a serem utilizados nos serviços públicos deve, obrigatoriamente,
ser pautada por esses princípios. Embora, todos estes princípios devam ser seguidos, nos
deteremos na eficiência a fim de demonstrar a pertinência ou não da utilização dos softwares livres
na administração pública.\cite{coutinho2012uso}

Como será explicado no capítulo Software livre, este tipo de software pode ser disponibilizado
gratuitamente. Além disso, pode ser modificado para prover melhorias ao
programa e ser redistribuído e copiado. Essa liberdade de melhorias, modificações e redistribuições
desses softwares é o principal diferencial entre softwares livres e softwares proprietários, já que
esses últimos não podem ser nem modificados, nem redistribuídos.

Sendo assim podemos apontar algumas razões para o uso de software livre na administração 
pública

\begin{itemize}

\item Nível de segurança proporcionado pelo Software Livre;
\item Eliminação de mudanças compulsórias que os modelos proprietários impõem
periodicamente a seus usuários, em face da descontinuidade de suporte a
versões ou soluções; 
\item Independência tecnológica; desenvolvimento de
conhecimento local; 
\item Possibilidade de auditabilidade dos sistemas;
\item Independência de fornecedor único;
\item Gratuidade das licenças de software livre.
\end{itemize}

Dessa forma este trabalho vem discutir os motivos pelos quais a administração 
federal não contrata software livre para uso nas entidades públicas mesmo citando 
os motivos acima.

Na Seção~\ref{software_livre} discutirei a respeito do software livre no brasil e no governo 
brasileiro bem como as licenças utilizadas nesses software e sobre o software público 
brasileiro,
%
na Seção~\ref{governanca} abordarei o conceito de governança e a teoria geral
da administração juntamente com o processo utilizado pelo SISP\footnote{http://www.sisp.gov.br/}
para desenvolvimento de software,
%
terminando com a Seção~\ref{consideracoes} onde esclarecerei a metodologia utilizada para desenvolvimento
deste trabalho, a hipótese levantada a partir do problema e as considerações e resultados preliminares 
deste trabalho além de um cronograma com os próximos passos. 


