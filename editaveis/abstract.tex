\begin{resumo}[Abstract]
 \begin{otherlanguage*}{english}
   
Free and public software are already established as a type of reliable software
due to the level of security provided by its use, eliminating changes
imposed by the proprietary model, technological independence, ability to
auditability beyond their gratuity allowances among other features.
%
In this work we try to understand the government's difficulties in using software
free/public addressing the effect of its licenses and restrictions on use and how they are
treated the copyright of these software in Brazil.
%
The objective of this work is to adapt the model of development used by
Brazilian government so that free software can be used and public
taking into account their good development practices.
%
To do a literature study will be conducted where the software
free and the public as well as the licenses used by these softwares and the way
with which they are applied. We will also software development processes
PSW-SISP and its origins in the theory of general administration looking for points of intersection
between this process and the method of community development and open source
public.

   \vspace{\onelineskip}
 
   \noindent 
   \textbf{Key-words}: free software. public software. governance.
 \end{otherlanguage*}
\end{resumo}
