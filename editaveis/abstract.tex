\begin{resumo}[Abstract]
 \begin{otherlanguage*}{english}
   
Public software projects are based on the principles and characteristics of open source projects as
to the level of security provided by its use, eliminating changes
imposed by the proprietary model, technological independence, ability to
auditability beyond their gratuity allowances among other features.
%
In this work we try to understand the difficulty of the developers to start
to contribute with public projects approaching the whole input process of
the developer in a community.
%
The objective of this work was check in context of public software the barriers
already verified in the literature and investigate specific barriers.
%
To comply this objetives we performed a reserch with public project communities
and we elaborate a new model of hurdling to developer public projects.

   \vspace{\onelineskip}
 
   \noindent 
   \textbf{Key-words}: free software. public software. hurdling. developer. communities.
 \end{otherlanguage*}
\end{resumo}
