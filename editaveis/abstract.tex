\begin{resumo}[Abstract]
 \begin{otherlanguage*}{english}
   
Free and public software are already established as a type of reliable software
due to the level of security provided by its use, eliminating changes
imposed by the proprietary model, technological independence, ability to
auditability beyond their gratuity allowances among other features.
%
In this work we try to understand the difficulty of the developers to start
to contribute with open/public projects approaching the whole input process of
the developer in a community.
%
The objective of this work is analyse and understand the hurdling to new developers
start the developer in open source projects, map the hurdling detected to for
developers to start in a public project, besides compare the new hurling with
knowledge hurdling.
%
To comply this objetives will be performed a reserch with public project communities
and will be elaborate a new model of hurdling to developer public projects and
from there to compare whit the model of hurdling to developer open source projects

   \vspace{\onelineskip}
 
   \noindent 
   \textbf{Key-words}: free software. public software. hurdling. developer. communities.
 \end{otherlanguage*}
\end{resumo}
