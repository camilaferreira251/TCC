\chapter{Considerações Preliminares}
\label{consideracoes}

Software livre/público já está consolidado como um tipo de software confiável 
devido ao nível de segurança proporcionado pelo seu uso, eliminação das mudanças
impostas pelo modelo proprietário, independência tecnológica, possibilidade de
auditabilidade além da gratuidade de suas licenças entre outras características.
%
Neste trabalho procuramos entender as dificuldades do governo em utilizar software
livre/público abordando o efeito de suas licenças e restrições de uso e como são 
tratado os direitos autorais desses softwares no Brasil.

Para investigar a hipótese levantada neste trabalho, a pesquisa bibliográfica 
apresentada estudou o processo de desenvolvimento utilizado pelo governo, também foi 
estudada a administração e suas teorias para entendermos a origem das metodologias
de desenvolvimento do governo e do software livre/público bem como o conceito de 
governança.

\section{Proposta}

O foco deste trabalho é adequar a metodologia de desenvolvimento do governo com a
metodologia utilizada pelas grandes comunidades de software livre a partir de boas 
práticas utilizadas por essas comunidades.
Está em desenvolvimento um documento com boas práticas operacionais do novo portal 
do software público e sua primeira versão está disponível no apêndice deste trabalho. 


