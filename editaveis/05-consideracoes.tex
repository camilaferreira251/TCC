\chapter{Considerações Preliminares}
\label{consideracoes}

Software livre/público já está consolidado como um tipo de software confiável 
devido ao nível de segurança proporcionado pelo seu uso, eliminação das mudanças
impostas pelo modelo proprietário, independência tecnológica, possibilidade de
auditabilidade além da gratuidade de suas licenças entre outras características.
%

\section{Metodologia de pesquisa}



\subsection{Abordagem Quantitativa versus Qualitativa}

As pesquisas, conforme as abordagens metodológicas que englobam, são classificadas em
dois grupos distintos – o quantitativo e o qualitativo. O primeiro obedece ao paradigma
clássico (positivismo) enquanto o outro segue o paradigma chamado alternativo~\cite{terence2006abordagem}.

\newpage
\begin{table}[htb]
\center
\footnotesize
\begin{tabular}{|p{4cm}|p{5cm}|p{5cm}|}
  \hline
   \textbf{} & \textbf{Pesquisa quantitativa}  & \textbf{Pesquisa qualitativa}\\
    \hline
   	Inferência & Dedutivo & Indutivo\\
   \hline    
    	Objetivo & Comprovação & Interpretação\\
   \hline
	Finalidade & Teste de teorias, predição, estabelecimento de fatos, e teste de hipóteses & Descrição e entendimento de realidades variadas, captura da vida cotidiana e perspectivas humanas\\
   \hline
	Realidade investigada & Objetiva & Subjetiva e complexa\\
   \hline
	Foco & Quantidade & Natureza do objeto\\
   \hline
	Amostra & Determinada por critério estatístico & Determinada por critérios diversos\\
   \hline
	Característica da amostra & Grande & Pequena\\
   \hline
	Característica do instrumento de coleta de dados & Questões objetivas, aplicações em um curto espaço de tempo & Questões abertas e flexíveis\\
   \hline
	Procedimentos & Isolamento de variáveis. Anônima aos participantes & Examina todo o contexto, interage com os participantes\\
   \hline
	Análise dos dados & Estatística e numérica & Interpretativa e descritiva. Ênfase na análise do conteúdo\\
   \hline
	Plano de pesquisa & Desenvolvido antes do estudo ser iniciado & Evolução de uma idéia como aprendizado\\
   \hline
	Resultados & Comprovação de hipóteses & Proposições e especulações\\
   \hline
	Confiabilidade e validade & Pode ser determinada dependendo do tempo e do recurso & Difícil determinação, dada a natureza subjetiva da pesquisa\\
   \hline
\end{tabular}
\caption{Características das abordagens qualitativa e quantitativa~\cite{terence2006abordagem}.}
\end{table}

Considerando as características apresentadas, a pesquisa desenvolvida neste trabalho
terá uma abordagem qualitativa por sua natureza indutiva, subjetiva e complexa, que será 
determinada por critérios diversos.

\section{Problema}

O Software Livre possui um mecanismo de produção colaborativo e dinâmico 
e possui uma organização composta por um conjunto de pessoas que usa e desenvolve 
um único software livre, contribuindo para uma base comum de código-fonte e 
conhecimento~\cite{reis2003caracterizacc}.

Este modelo típico do Software livre se diferencia em muitos aspectos com a forma
que o governo brasileiro desenvolve software, onde se estabelece um rígido processo.


\section{Hipótese}

As boas práticas de desenvolvimento a serem utilizadas em um projeto de software 
dependem do contexto, uma equipe que desenvolve software livre utiliza
meios diferentes para gerenciar o próprio projeto, que pode conter especificidades 
que impedem o uso de determinadas práticas impostas pelo governo federal para controle
do desenvolvimento de software. 

Dessa forma, com base no problema proposto, elaboramos a seguinte hipótese:

\begin{itemize}
\item \emph{H1::}

\end{itemize}


