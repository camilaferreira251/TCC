\chapter{Considerações Preliminares}
\label{consideracoes}

Software livre/público já está consolidado como um tipo de software confiável 
devido ao nível de segurança proporcionado pelo seu uso, eliminação das mudanças
impostas pelo modelo proprietário, independência tecnológica, possibilidade de
auditabilidade além da gratuidade de suas licenças entre outras características.
%
Neste trabalho procuramos entender as dificuldades do governo em utilizar software
livre/público abordando o efeito de suas licenças e restrições de uso e como são 
tratado os direitos autorais desses softwares no Brasil.

Para investigar a hipótese levantada neste trabalho, a pesquisa bibliográfica 
apresentada estudou o processo de desenvolvimento utilizado pelo governo, também foi 
estudada a administração e suas teorias para entendermos a origem das metodologias
de desenvolvimento do governo e do software livre/público bem como o conceito de 
governança.

Com este trabalho foi possível notar a que a diferenciação das metodologias estão
na origem das mesmas tendo como base a teoria geral da administração, onde pude reparar
que a teoria humanística utilizada no desenvolvimento de software livre/público se
opõe a teoria de Taylor que tem suas bases nos processos e é utilizada pelo governo.

\section{Proposta}

O foco deste trabalho é adequar a metodologia de desenvolvimento do governo com a
metodologia utilizada pelas grandes comunidades de software livre a partir de boas 
práticas utilizadas por essas comunidades.
Está em desenvolvimento um documento com boas práticas operacionais do novo portal 
do software público e sua primeira versão está disponível no apêndice deste trabalho. 

\subsection{Próximos passos}

Para alcançar os objetivos deste trabalho, serão realizadas as seguinte macro-atividades
na segunda etapa deste trabalho. Seguindo a metodologia proposta utilizarei os conhecimentos
adquiridos na pesquisa desta etapa do trabalho para apresentar uma ação para resolver o 
problema segundo a tabela~\ref{cronograma}.

\newpage

\begin{table}[htb]
\center
\footnotesize
\begin{tabular}{|p{8cm}|p{3cm}|p{3cm}|}
  \hline
   \textbf{Atividade} & \textbf{Início} & \textbf{Fim}\\
    \hline
	Aprofundar estudo de metodologia de desenvolvimento do governo & 01/02/2015 & 01/03/2015\\
    \hline
	Aprofundar estudo de metodologia de desenvolvimento software livre/público & 01/02/2015 & 01/03/2015\\
    \hline
        Caracterizar os processos de desenvolvimento buscando pontos de intersecção & 02/03/2015 & 15/03/2015\\
    \hline
	Coletar experiência e boas práticas dos usuários das metodologias & 16/03/2015 & 01/04/2015\\
    \hline
	Analisar dados coletados & 02/04/2015 & 15/04/2015\\
    \hline 
	Discutir resultados encontrados & 16/04/2015 & 01/05/2015\\
    \hline
	Montar governança adequada com base nos resultados & 02/05/2015 & 01/06/2015\\
    \hline
	Discutir governança proposta & 02/06/2015 & 20/06/2015\\
    \hline
	Entrega do TCC2 & 20/06/2015 & 30/06/2015\\
    \hline
\end{tabular}
\caption{Cronograma para o TCC2}
\label{cronograma}
\end{table}

As primeiras atividades de aprofundamento de estudo das metodologias serão 
necessárias que ocorram paralelamente para que já seja buscada uma comparação 
entre as mesmas e estas atividades darão subsídios para que sejam caracterizados 
os dois processos de desenvolvimento.
%
Coletar experiência dos usuários será uma maneira de observar as boas práticas de cada 
metodologia e com a análise das experiência poderei distinguir o que fica e o que 
é necessário que seja repensado em cada processo.
%
A metodologia proposta será montada a partir dessas experiências e estudos relacionados
trazendo embasamento para a metodologia proposta. 
