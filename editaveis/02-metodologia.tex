\chapter{Metodologia}

Neste capítulo será apresentada a metodologia de pesquisa adotada para desenvolvimento 
deste trabalho de conclusão de curso bem como o problema a ser resolvido por essa pesquisa.

\section{Pesquisa-ação}

A pesquisa-ação ou action-research surgiu na década de 1940, no contexto de críticas ao uso
de procedimentos clássicos de ciências naturais na pesquisa social por razões de ordem prática 
(conhecimento teórico gerado teria pouca aplicabilidade na prática) ou ideológica (pesquisas estariam
sendo realizadas como uma forma de controle social)\cite{gil2010metodos}.

O pesquisador na pesquisa-ação
assume como premissa que processos sociais complexos, como a interação
entre organizações e seus sistemas de informação, são melhor investigados
quando se introduzem mudanças nestes processos e se observa os efeitos
destas mudanças. Outra premissa é que estes processos devem ser
investigados como uma entidade completa, não sendo possível extrair o objeto
de investigação do seu contexto. Desta forma, na pesquisa-ação busca-se avançar na teoria 
atuando na prática, o que é feito através de
ações no contexto de uma organização específica. O foco do pesquisador é na
compreensão do problema e das ações realizadas para solucioná-lo dentro de
um ambiente real particular e não na verificação de uma hipótese de caráter
geral num ambiente de laboratório~\cite{fuks2008suporte}.

Segundo Ana Paula~\cite{fuks2008suporte}"O pesquisador pode ter uma visão de \emph{insider}
ou \emph{outsider}, na realização de uma pesquisa-ação.
A visão de um \emph{insider} ocorre quando o pesquisador vivencia o problema e o traz para a pesquisa,
ou seja leva seus problemas para realizar a pesquisa, já a visão de outsider é quando a instituição
tem um problema e chama um pesquisador ou o pesquisador vai atrás de uma empresa ou cenário
real que tenha o problema. No caso deste estudo, a pesquisa-ação realizada no projeto Arquigrafia-
Brasil, é do tipo \emph{outsider}, pois existia a necessidade de pesquisa de usabilidade no projeto em um
contexto de desenvolvimento de software livre utilizando métodos ágeis, que consiste no cenário
ideal para esta pesquisa". Com base na definição apresentada é possível afirmar que a pesquisa-ação a ser 
apresentada é \emph{insider}

\section{Problema}

O Software Livre possui um mecanismo de produção colaborativo e dinâmico 
e possui uma organização composta por um conjunto de pessoas que usa e desenvolve 
um único software livre, contribuindo para uma base comum de código-fonte e 
conhecimento.\cite{reis2003caracterizacc}

Este modelo típico do Software livre se diferencia em muitos aspectos com a forma
que o governo brasileiro desenvolve software, onde se estabelece um rígido processo.

Apesar disso, por fatores sociais e econômicos o governo brasileiro vem
criando políticas de incentivo a software livre, mas a dificuldade está em como
fazer a governança de dois mecanismos de gestão da produção de software antagônicos. 	




