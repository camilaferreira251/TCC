\chapter[Software Livre]{Software Livre}
\label{software_livre}

O software livre muitas vezes é considerado um fenômeno recente que veio à tona
rapidamente nos últimos anos. No entanto, desde o início da Computação a maior parte
dos desenvolvedores trabalhava da forma que hoje identificamos com o software livre:
compartilhando código de forma aberta. Essa característica 
faz com que o código esteja disponível para inspeção, alteração, e utilização por qualquer
pessoas, física ou jurídica~\cite{kon2012software, Hippel:2003:OSS:970521.970585}.

Software representa uma solução abstrata para problemas do mundo real baseados
em um domínio de conhecimento específico e por isso tem interessados de 
diversas áreas. As características relacionadas a ele vão além de questões
técnicas, é necessário entender o processo de desenvolvimento, o contexto econômico,
a relação entre os desenvolvedores e usuários além das características éticas e
legais que se relacionam ao software.
%
Os elementos que diferenciam o software livre de outros tipos de software 
é o entendimento dos atributos apresentados acima. No contexto do software livre
é promovida a liberdade do usuário, seja qual for seu tipo e seu desenvolvimento
tem alicerce em colaboração e desenvolvimento aberto. Dessa forma todos podem
estudar, modificar e redistribuir o software, sem impedimentos legais aos 
futuros usuários~\cite{meirelles2013metrics}.

Do ponto de vista econômico, diferentemente do que ocorre com o software 
restrito, o software livre promove o estabelecimento
de vários fornecedores que competem entre si com base no mesmo software. Essa competição 
mais forte entre fornecedores traz vantagens para os usuários, pois dá melhores
garantias quanto ao desenvolvimento futuro do sistema e induz a uma redução nos 
preços. Essas liberdades e garantias sobre o software são estabelecidas 
no Brasil pela lei de direitos autorais~\cite{kon2012software}. 

Os projetos de software livre seguem o estilo Linus Trovalds de desenvolvimento, libere
cedo e frequentemente, delegue tudo que você possa, esteja aberto ao ponto de promiscuidade.
O desenvolviemtno de software não pode se assemelhar a uma catedral calma e respeitosa, e sim, 
parecer a um grande e barulhento bazar de diferentes agendas e aproximações. 
Em uma cultura que encoraja a troca de software, isto é, um caminho natural para o software evoluir.
A principal diferença entre o estilo Catedral e Bazar é o estilo de programação, 
erros e problemas de desenvolvimento são difíceis para o estilo Catedral, leva meses 
de exame minuncioso de poucas pessoas dedicadas e longos intervalos de liberação 
de novos trechos de código.
Na visão Bazar, erros são geralmente um fenômeno trivial, pois é exposto a centenas de 
desenvolvedores com liberação frenquente de código~\cite{Raymond:1999:CB:580808}.

Quando você começa a contrução de uma comunidade, seu programa não precisa funcionar
particularmente bem, o que não pode deixar de fazer é convencer os prováveis 
desenvolvedores de que pode ser evoluído para algo elegante em um futuro próximo
~\cite{Raymond:1999:CB:580808}.

O uso de ferramentas de controle de versão se tornou fundamental para o gerenciamento
do desenvolvimento de projetos de software livre. Uma vez que seu projeto está 
disponível em uma dessas ferramentas, um desenvolvedor pode acessá-lo e criar
uma cópia daquele código sobre os mesmos termos da licensa, explicada abaixo, e 
colaborar com o desenvolvimento. Para essa colaboração, os desenvolvedores criam
\textit{issues}, que são tarefas, essas tarefas são desenvolvidas em seu ambiente
e quando o desenvolvedor as termina, envia para ser mesclado ao código oficial através de um 
\textit{merge request}, onde o código será revisado e pode ou não ser aceito.


\section{Lei de direitos autorais}

Boa parte do software atualmente usado e desenvolvido, seja para computadores 
pessoais ou servidores, seja para uso geral ou específico, é disponibilizado sob
licenças restritivas. Essas licenças, em maior ou menor grau, impõem restrições 
ao seu uso, distribuição ou acesso ao código-fonte. Esse tipo de licenciamento é 
possível porque o software está sujeito à proteção da lei a respeito dos direitos 
de autor, que garante ao criador o direito exclusivo de exploração de sua obra. 
Isso lhe permite autorizar ou não determinadas formas de uso do software por parte 
dos usuários.
A legislação brasileira vê o software menos como produto e mais como expressão 
intelectual, prevendo que os direitos de autor são o mecanismo próprio de proteção 
ao software e excluindo explicitamente patentes como opção (Lei 9609/98 e Lei 9279/98, art. 10)
\cite{softwarepublico}.

Segundo o Instituto Nacional de Tecnologia da Informação\footnote{http://www.iti.gov.br/}, 
os direitos autorais sobre software, assim como sobre as obras literárias, são 
independentes de registro. A Lei do Software estabelece, no entanto, que o INPI\footnote{http://www.inpi.gov.br}
é o órgão governamental encarregado do registro do software. O registro serve 
como comprovação de anterioridade de autoria sobre o software caso esta venha 
em algum momento a ser questionada judicialmente.

A legislação autoral e de Propriedade Intelectual reconhece o software como objeto 
de direito do autor, com regime específico dado pela lei do software (Lei nº 9609/98)
subsidiada naquilo que for omissa pela Lei do Direito Autoral (Lei nº 9610/98).
%
Assim, cabe ao titular do direito autoral a definição da forma como disporá desse
direito, se em regime proprietário, ou em regime livre. Essa decisão, certamente,
situa-se fora do âmbito de atuação do estado~\cite{junior2005software}.

Segundo estudo sobre o software livre, comissionado pelo instituto nacional da 
tecnologia da informação (ITI), o software no Brasil é regido pelo direito autoral. 
Na sua maioria das vezes, essa proteção decorrente da lei segue aliada aos termos 
conferidos por um contrato atinente a determinado software. Esse contrato é 
denominado ``licença''. A licença de um software estabelece um rol de direitos e 
deveres que se projetam sobre um determinado usuário do software. Em princípio, 
o que diferencia o software livre do software restrito é apenas a forma de
licenciamento~\cite{kon2012software}.


\section{Licenças de Software}

Programas de software livre, em geral, são de fácil acesso. Porém, a simples obtenção
de um programa não significa que a pessoa pode fazer o que quiser com ele. As licenças
de software livre são documentos através dos quais os detentores dos direitos sobre um
programa de computador autorizam usos de seu trabalho que, de outra forma, estariam
protegidos pelas leis vigentes no local~\cite{sabino2009licenccas}.

Para que um software seja dito livre aquele que tem os direitos sobre o software
deve aplicar ao mesmo uma licensa que o caracterize como software livre. Dessa 
forma, todos aqueles que desenvolvem o software deverão fazê-los sobre os termos
da licensa que deve ser apresentada junto com o código fonte~\cite{sabino2009licenccas}.

Software livre pode ser distribuído em uma grande variedade de licenças.
A mais famosa é a GNU General Public License (GPL), usada pela maior
parte dos softwares e foi desenvolvida pela Free Software Foundation \footnote{https://www.fsf.org/pt-br}.

As licenças de software são separadas em permissivas, recíprocas totais e recíprocas
parciais.
%
As licenças permissivas impõe poucas restrições ao uso e compartilhamento do
software, não é feita nenhuma restrição ao licenciamento de trabalhos derivados, 
estes podendo inclusive serem distribuídos sob uma licença fechada. Licenças permissivas são 
uma ótima opção para projetos cujo objetivo é atingir o maior número de pessoas, 
não importando se na forma de software livre ou de software fechado. As principais 
licenças dessa categoria são: BSD\footnote{www.creativecommons.org/licenses/BSD/legalcode},
MIT/X11\footnote{www.opensource.org/licenses/mit-license.php} e Apache\footnote{www.apache.org/licenses/LICENSE-2.0.html}
~\cite{sabino2009licenccas}.

As licenças recíprocas totais determinam que qualquer trabalho derivado precisa 
ser distribuído sob os mesmos termos da licença original. Isso também é chamado de copyleft,
um termo criado pela Free Software Foundation\footnote{www.gnu.org/copyleft}. A idéia do
copyleft é dar permissão a todos para executar, copiar, modificar e distribuir versões
modificadas do programa, mas impedir que sejam adicionadas restrições a essas versões
redistribuídas. A licença que deu origem à idéia de copyleft foi a General Public 
License, ou GPL\footnote{www.gnu.org/licenses/gpl.html}, da Free Software Foundation.
As principais licensas dassa categoria são as devivadas da GPL: 
GPL2.0\footnote{http://www.gnu.org/licenses/old-licenses/gpl-1.0.html}, 
GPLv3\footnote{www.gnu.org/licenses/gpl-3.0-standalone.html} e a
AGPL\footnote{www.affero.org/oagpl.html},que foi desenvolvida pela empresa 
Affero\footnote{www.affero.org}.~\cite{sabino2009licenccas}

Licenças recíprocas parciais, também chamadas de copyleft fraco, determinam que
modificações do trabalho coberto por elas devem ser disponibilizadas sob a mesma 
licença. Porém, quando o trabalho é utilizado apenas como um componente de outro 
projeto, esse projeto não precisa estar sob a mesma licença. As principais licenças
dessa categoria são: LGPL\footnote{www.fsf.org/licensing/licenses/lgpl.html} e a
Licença Mozilla~\cite{sabino2009licenccas}.

Para que um software se torne software público hoje, entre outras coisas, ele deve
possuir licença que garanta ao receptor do software: executar, estudar, 
adaptar conforme as suas necessidades, redistribuir e aperfeiçoar o software. 

\section{Software Público}

O Portal do Software Público Brasileiro - SPB, inaugurado em 2007, na prática, é um sistema
web que se consolidou como um ambiente de compartilhamento de projetos de software. Oferece um
espaço (comunidade) para cada software. Por isso, a nova plataforma para o SPB foi pensada
para contemplar ferramentas que promovam a colaboração e a interação nas comunidades 
(por gestores, usuários e desenvolvedores) dos projetos, conforme as práticas usadas nas
comunidades de software livre. Isso inclui listas de e-mail, fóruns de discussão, 
issue trackers, sistemas de controle de versão e ambientes de rede social.
O SPB teve um crescimento expressivo contando, hoje, com mais de 70 comunidades
de desenvolvimento e mais de 200.000 usuários cadastrados. O SPB abrange também o 
AvaliaSPB que avalia a entrada dos softwares candidatos à software público~\cite{softwarepublico}.

Inicialmente, o propósito do Portal de compartilhar os softwares desenvolvidos
no governo visando diminuir os custos de contratação de software, mas se observou
que ao disponibilizar os softwares rapidamente se formavam comunidades em torno 
daquele software com diversas pessoas colaborando e compartilhando os resultados
obtidos através do uso daquelas soluções. Desta forma algumas cooperativas de 
desenvolvimento de software e empresas privados demonstraram o interesse em
publicar seus softwares na plataforma que estava sendo criada.

O Governo Federal criou em 2005 o modelo do Software
Público Brasileiro (SPB), que entre os usuários estão ofertantes e demandantes de soluções, 
organizados em comunidades, criadas em torno de cada solução de software. A intensidade de participação
varia desde um observador interessado no software até líderes de comunidades. Essa diversidade é
derivada do modelo de produção do software livre, no qual baseou-se o SPB para sua formação
~\cite{alves2009software}. 

A percepção do potencial que representava a participação da sociedade no desenvolvimento do software
e o conceito de bem público foram adaptadas do ponto de vista jurídico, assim levando o Ministério
do Planejamento, Orçamento e Gestão (MP) a formular o conceito de software público. Essa base
jurídico-institucional permitiu a criação de um ambiente virtual (um portal) para a disponibilização
de software como software público. Esse modelo é definido por uma rede que se auto-organiza e cuja
produção se caracteriza pela intensa participação colaborativa entre indivíduos, empresa ou 
prestadores de serviço, universidades e instituições interessadas na evolução de um determinado projeto
de software~\cite{alves2009software}.

O conceito de software público diferencia-se do de software livre em alguns aspectos, destacando-se
a atribuição de bem público ao software. Embora haja algumas diferenças entre o que é um software
livre e um software público brasileiro, há princípios comuns, como a tendência da descentralização na
tomada de decisões, o intenso compartilhamento de informações e os processos de retroalimentação
decorrentes do uso dos artefatos produzidos. 

A Instrução Normativa 04/2012 indica que os gestores devem consultar as soluções 
existentes no portal do SPB antes de realizar uma contratação de software, apesar de
a evolução técnica do portal do SPB ficar comprometida desde 2009, ao não acompanhar 
a evolução do seu framework base, o OpenACS. Dessa forma não houve o lançamento de novas versões do
portal desde então\cite{softwarepublico}.

Uma nova plataforma para o SPB foi desenvolvida
pela Universidade de Brasília, através dos seus Laboratórios LAPPIS\footnote{https://fga.unb.br/lappis} 
e MídiaLab\footnote{http://www.midialab.unb.br/} 
em parceria com o Centro de Competência em Software Livre da Universidade de São Paulo(CCSL-USP)\footnote{http://ccsl.ime.usp.br/}.

A nova plataforma do SPB é composta por um conjunto de ferramentas com diferentes funcionalidades
que juntas compõe o novo Portal. Todas as ferramentas são software livre e o que foi desenvolvido pelas equipes da UnB e 
USP foi publicado em repositórios abertos, disponíveis no próprio SPB, as principais são:
\begin{itemize}

\item Colab, ferramenta desenvolvida pela Interlegis\footnote{http://www.interlegis.leg.br/}, órgão ligado 
ao Senado Federal, que têm como objetivo principal a integração com outras ferramentas.

\item Noosfero, é um plataforma de criação de redes sociais que conta com as funcionalidades
que uma rede social provê, bem como blogs, galeria de imagens, entre outras. É mantido pela 
COOLIVRE\footnote{http://colivre.coop.br/}, pelo SERPRO\footnote{https://www.serpro.gov.br/} e pela
Universidade de Brasília.

\item Gitlab, ferramenta de desenvolvimento colaborativo, projetos no gitlab são mantidos em 
repositório Git que provê: \textit{issue tracker}, \textit{merge requests}, 
\textit{milestones}, suporte a integração com plataformas de integração contínua e notificações.

\item GNU Mailman, ferramenta que gerencia listas de e-mail.

\end{itemize}

A arquitetura do novo Portal pode ser vizualizada a seguir\footnote{https://softwarepublico.gov.br/doc/arquitetura.html}.

\begin{figure}[h]
	\centering
	\label{arquitetura}
		\includegraphics[keepaspectratio=true,scale=0.3]{figuras/arquitetura.eps}
	\caption{Arquitetura do novo SPB}
\end{figure}


A nova plataforma do SPB foi lançada em 2015 e já está sendo utilizada pelas comunidades
do SPB, a antiga versão do Portal já foi desativada e suas comunidades foram 
migradas para a nova plataforma. 

Disponibilizar um conjunto de ferramentas e melhorar a experiência do usuário no 
ambiente é parte desse processo de reformulação do SPB. Aspectos culturais 
da colaboração em rede para um efetivo uso do que é fornecido na plataforma 
necessitam ser amadurecidos pelo MPOG junto às comunidades do SPB.

A Portaria Nº 46, de 28 de Setembro de 2016, definiu que software público nada mais é
que um software livre que atende às necessidades de modernização da administração 
pública além de identificar os papéis de colaborador, para quem contribui com os projetos do SPB, 
coordenador de comunidade, para quem tem a responsabilidade de interação com a comunidade e 
ofertante de software para quem disponibiliza o software. As comunidades passam a poder
ser abertas para todos os usuários SPB ou moderadas, onde estarão presentes apenas os 
usuários adicionados por seu coordenador. Passam a fazer parte do SPB também os 
softwares de governo, softwares desenvolvidos por um órgão do governo que precisa ser
compartilhado mas não atende a todos os requisitos para ser software público, além de projeto de
software que é um novo projeto que utiliza da infraestrutura do SPB mas ainda não
foi definido como software público.
% 
Também esta presente nessa Portaria a atualização das
regras para oferta de software no Portal no SPB como exemplo da flexibilização
da lincensa de software aceita e a não mais obrigatoriedade do registro do 
software no INPI. 
%
A fim de que os software que se encontrem no SPB não fique desatualizados perante
o Catálogo de Software do SISP, a Portaria Nº 48, de 28 de Setembro de 2016, definiu que 
as informações a respeito das soluções sejam atulaizadas anualmente.

Este trabalho tem o objetivo de ajudar a entender as dificuldades enfrentadas
pelas comunidades, buscando investigar quais as barreiras que o desenvolvedor 
enfrenta para contribuir com os projetos. Tendo essas barreiras será possível
criar propostas de guias e treinamentos para mitigar os entraves das comunidades.  

