\chapter[Software Livre]{Software Livre}
\label{software_livre}

Neste capítulo iniciaremos introduzindo o assunto sobre software livre e como ele é
tratado no governo brasileiro pela lei de direitos autorais, licensas de software e 
a solução proposta pelo governo chamada software público brasileiro. 

O software livre muitas vezes é considerado um fenômeno recente que veio à tona
rapidamente nos últimos anos. No entanto, desde o início da Computação a maior parte
dos desenvolvedores trabalhava da forma que hoje identificamos com o software livre:
compartilhando código de forma aberta. Essa característica 
faz com que o código esteja disponível para inspeção, alteração, e utilização por qualquer
pessoas, física ou jurídica~\cite{kon2012software, Hippel:2003:OSS:970521.970585}.

%TODO {
Software expressa uma solução abstrata dos problemas computacionais. O software, em um
sistema computacional, é o componente que contém o conhecimento relacionado aos problemas a
que a computação se aplica. Por isso, o software é algo de interesse geral, uma vez que vários
aspectos relacionados a ele ultrapassam as questões técnicas, como por exemplo:

\begin{itemize}

\item O processo de desenvolvimento do software;

\item Os mecanismos econômicos (gerenciais, competitivos, sociais, cognitivos etc.) que regem esse
desenvolvimento e seu uso;

\item O relacionamento entre desenvolvedores, fornecedores e usuários do software;

\item Os aspectos éticos e legais relacionados ao software.

\end{itemize}

O que define e diferencia o software livre do que podemos denominar de software restrito passa
pelo entendimento desses quatro pontos dentro do que é conhecido como o ecossistema do 
software livre. O princípio básico desse ecossistema é promover a liberdade do 
usuário, sem discriminar quem tem permissão para usar um software e seus limites 
de uso, baseado na colaboração e num processo de desenvolvimento aberto. Software 
livre é aquele que permite aos usuários usá-lo, estudá-lo, modificá-lo e redistribui-lo, 
em geral, sem restrições para tal e prevenindo que não sejam impostas restrições aos 
futuros usuários~\cite{meirelles2013metrics}.
%TODO: Reescreve com suas palavras, uma vez que é TCC 2}


Do ponto de vista econômico, diferentemente do que ocorre com o software 
restrito, o software livre promove o estabelecimento
de vários fornecedores que competem entre si com base no mesmo software. Essa competição 
mais forte entre fornecedores traz vantagens para os usuários, pois dá melhores
garantias quanto ao desenvolvimento futuro do sistema e induz a uma redução nos 
preços~\cite{kon2012software}.

Os projetos de software livre seguem o estilo Linus Trovalds de desenvolvimento, libere
cedo e frequentemente, delegue tudo que você possa, esteja aberto ao ponto de promiscuidade.
Nenhuma catedral calma e respeitosa, e sim, parecer assemelhar-se a um grande e barulhento 
bazar de diferentes agendas e aproximações. Em uma cultura que encoraja a troca de
software, isto é, um caminho natural para o software evoluir.

%TODO:
A principal diferença 
entre o estilo Catedral e Baazar é a programação, erros e problemas de desenvolvimento 
são difíceis, leva meses de exame minuncioso de poucas pessoas dedicadas e longos 
intervalos de liberação de novos trechos de código no estilo catedral.
%TODO: está confuso, talvez por erro de pontuação

Na visão 
Baazar, erros são geralmente um fenômeno trivial, pois é exposto a centenas de 
desenvolvedores com liberação frenquente de código~\cite{Raymond:1999:CB:580808}.

Quando você começa a contrução de uma comunidade, seu programa não precisa funcionar
particularmente bem, o que não pode deixar de fazer é convencer os prováveis 
desenvolvedores de que pode ser evoluído para algo elegante em um futuro próximo
~\cite{Raymond:1999:CB:580808}.

O uso de ferramentas de controle de versão se tornou fundamental para o gerenciamento
do desenvolvimento de projetos de software livre. Uma vez que seu projeto está 
disponível em uma dessas ferramentas, um desenvolvedor pode acessá-lo e criar
uma cópia daquele código sobre os mesmos termos da licensa, explicada abaixo, e 
colaborar com o desenvolvimento. Para essa colaboração, os desenvolvesores criam
\textit{issues}, que são tarefas, essas tarefas são desenvolvidas em seu ambiente
e quando o desenolvedor as termina e enviado para o código oficial por um 
\textit{merge request} onde o código será revisado e pode ou não ser aceito.

Estima-se que hoje há centenas de milhões de usuários de software livre no mundo. Se
considerarmos também usuários indiretos, que usam serviços baseados em software livre,
como Google, Amazon ou eBay, esse número é ainda maior~\cite{sabino2009licenccas}.

O governo vêm adotando um conjunto de iniciativas para implementação de software
livre no Brasil como reação aos enormes custos gerados à Administração com licenças
de softwares, o Decreto presidencial 18/00 instituiu o Comitê Executivo do Governo
Eletrônico, no intuito de racionalizar gastos com softwares. Para efetivar esse 
objetivo, instituiu o Comitê Técnico para Implementação de Software Livre, por
meio do Decreto nº 29/03.
%Governo Temer acabou com o CISL - Temos atualizar o tempo aqui

A lei que rege sobre o software no Brasil é a Lei de Direitos Autorais e será 
explicada em detalhes a seguir. 


\section{Lei de direitos autorais}

Boa parte do software atualmente usado e desenvolvido, seja para computadores 
pessoais ou servidores, seja para uso geral ou específico, é disponibilizado sob
licenças restritivas. Essas licenças, em maior ou menor grau, impõem restrições 
ao seu uso, distribuição ou acesso ao código-fonte. Esse tipo de licenciamento é 
possível porque o software está sujeito à proteção da lei a respeito dos direitos 
de autor, que garante ao criador o direito exclusivo de exploração de sua obra. 
Isso lhe permite autorizar ou não determinadas formas de uso do software por parte 
dos usuários.%artigo paulo
%TODO: referências

A legislação brasileira vê o software menos como produto e mais como expressão 
intelectual, prevendo que os direitos de autor são o mecanismo próprio de proteção 
ao software e excluindo explicitamente patentes como opção (Lei 9609/98 e Lei 9279/98, art. 10).
%TODO: referência
Segundo o instituto nacional de tecnologia da informação\footnote{http://www.iti.gov.br/}
, os direitos autorais sobre software, assim como sobre as obras literárias, são 
independentes de registro. A Lei do Software estabelece, no entanto, que o INPI\footnote{http://www.inpi.gov.br}
é o órgão governamental encarregado do registro do software. O registro serve 
como comprovação de anterioridade de autoria sobre o software caso esta venha 
em algum momento a ser questionada judicialmente.

A legislação autoral e de Propriedade Intelectual reconhece o software como objeto 
de direito do autor, com regime específico dado pela lei do software (Lei nº 9609/98)
subsidiada naquilo que for omissa pela Lei do Direito Autoral (Lei nº 9610/98).
%
Assim, cabe ao titular do direito autoral a definição da forma como disporá desse
direito, se em regime proprietário, ou em regime livre. Essa decisão, certamente,
situa-se fora do âmbito de atuação do estado~\cite{junior2005software}.

Segundo estudo sobre o software livre, comissionado pelo instituto nacional da 
tecnologia da informação (ITI), o software no Brasil é regido pelo direito autoral. 
Na sua maioria das vezes, essa proteção decorrente da lei segue aliada aos termos 
conferidos por um contrato atinente a determinado software. Esse contrato é 
denominado ``licença''. A licença de um software estabelece um rol de direitos e 
deveres que se projetam sobre um determinado usuário do software. Em princípio, 
o que diferencia o software livre do software restrito é apenas a forma de
licenciamento. \cite{kon2012software}

A produção de software livre foi inicialmente concebida como
uma alternativa ao modelo proprietário tradicional, onde o uso
do aplicativo era vendido sob os termos de uma licença. \footnote{http://sbc.org.br/component/flippingbook/book/21}
%TODO: o que esse footnote?


\section{Licenças de Software}

Programas de software livre, em geral, são de fácil acesso. Porém, a simples obtenção
de um programa não significa que a pessoa pode fazer o que quiser com ele. As licenças
de software livre são documentos através dos quais os detentores dos direitos sobre um
programa de computador autorizam usos de seu trabalho que, de outra forma, estariam
protegidos pelas leis vigentes no local~\cite{sabino2009licenccas}.

Para que um software seja dito livre ``o detentor dos direitos patrimoniais 
sobre um software, deve escolher os termos em que seu trabalho será distribuído, 
ou seja, os direitos que ele estará transferindo para as outras pessoas e quais as 
condições que serão aplicadas. O documento que formaliza esse ato é a licença, 
que normalmente é distribuída junto com o código fonte''~\cite{sabino2009licenccas}.
%TODO: tentar evitar citação direta, entre aspas; ou seja, escrever com suas palavras

Software aberto pode ser distribuído em uma grande variedade de licenças.
A mais famosa é a GNU General Public License (GPL), usada pela maior
parte do software desenvolvido pela Free Software Foundation \footnote{https://www.fsf.org/pt-br}.
%TODO: aqui vamos usar apenas software livre e não aberto

As licenças de software são separadas em permissivas, recíprocas totais e recíprocas
parciais.

As licenças permissivas impõe poucas restrições ao uso e compartilhamento do
software, não é feita nenhuma restrição ao licenciamento de trabalhos derivados, 
estes podendo inclusive serem distribuídos sob uma licença fechada. Licenças permissivas são 
uma ótima opção para projetos cujo objetivo é atingir o maior número de pessoas, 
não importando se na forma de software livre ou de software fechado. As principais 
licenças dessa categoria são: BSD\footnote{www.creativecommons.org/licenses/BSD/legalcode},
MIT/X11\footnote{www.opensource.org/licenses/mit-license.php} e Apache\footnote{www.apache.org/licenses/LICENSE-2.0.html}
~\cite{sabino2009licenccas}.

As licenças recíprocas totais determinam que qualquer trabalho derivado precisa 
ser distribuído sob os mesmos termos da licença original. Isso também é chamado de copyleft,
um termo criado pela Free Software Foundation\footnote{www.gnu.org/copyleft}. A idéia do
copyleft é dar permissão a todos para executar, copiar, modificar e distribuir versões
modificadas do programa, mas impedir que sejam adicionadas restrições a essas versões
redistribuídas. A licença que deu origem à idéia de copyleft foi a General Public 
License, ou GPL\footnote{www.gnu.org/licenses/gpl.html}, da Free Software Foundation.
As principais licensas dassa categoria são as devivadas da GPL: 
GPL2.0\footnote{http://www.gnu.org/licenses/old-licenses/gpl-1.0.html}, 
GPLv3\footnote{www.gnu.org/licenses/gpl-3.0-standalone.html} e a
AGPL\footnote{www.affero.org/oagpl.html},que foi desenvolvida pela empresa 
Affero\footnote{www.affero.org}.~\cite{sabino2009licenccas}

Licenças recíprocas parciais, também chamadas de copyleft fraco, determinam que
modificações do trabalho coberto por elas devem ser disponibilizadas sob a mesma 
licença. Porém, quando o trabalho é utilizado apenas como um componente de outro 
projeto, esse projeto não precisa estar sob a mesma licença. As principais licenças
dessa categoria são: LGPL\footnote{www.fsf.org/licensing/licenses/lgpl.html} e a
Licença Mozilla~\cite{sabino2009licenccas}.

Para que um software se torne software público hoje, entre outras coisas, ele deve
possuir licença GPL2.0, conforme Instrução Normativa nº 01 de 17 de janeiro de 2011.
%TODO: terá que falar da nova IN que flexibilizou isso   

\section{Software Público}

O Portal do Software Público Brasileiro - SPB, inaugurado em 2007, na prática, é um sistema
web que se consolidou como um ambiente de compartilhamento de projetos de software. Oferece um
espaço (comunidade) para cada software. A comunidade é composta por fórum, notícias, chat, 
armazenamento de arquivos e downloads, wiki, lista de prestadores de serviços, usuários, coordenadores,
entre outros recursos. Teve um crescimento expressivo contando, hoje, com mais de 60 comunidades
de desenvolvimento e mais de 200.000 usuários cadastrados. O SPB abrange também, o 4CMBr que
é o grupo de interesse voltado para soluções de tecnologia para municípios, o 5CQualiBr que é um
grupo que trabalha para evoluir a qualidade do Software Público Brasileiro, o 4CTecBr, um portal
destinado a colaboração no desenvolvimento de Tecnologias Livres, o Mercado Público Virtual que
é um grupo de empresas e pessoas que prestam serviço nos softwares ofertados no Portal e o 
AvaliaSPB que avalia a entrada dos softwares candidatos à software público. O ambiente do SPB não
proporciona a integração com ambientes colaborativos externos, especialmente com redes sociais. A
plataforma escolhida na ocasião da criação foi o framework OpenACS, que continua sendo utilizada
na atual versão.
%TODO: atualizar as informações

Inicialmente, o propósito do Portal de compartilhar os softwares desenvolvidos
no governo visando diminuir os custos de contratação de software, mas se observou
que ao disponibilizar os softwares rapidamente se formavam comunidades em torno 
daquele software com diversas pessoas colaborando e compartilhando os resultados
obtidos através do uso daquelas soluções. Desta forma algumas cooperativas de 
desenvolvimento de software e empresas privados demonstraram o interesse em
publicar seus softwares na plataforma que estava sendo criada.

O Governo Federal criou em 2005 o modelo do Software
Público Brasileiro (SPB), que entre os usuários estão ofertantes e demandantes de soluções, 
organizados em comunidades, criadas em torno de cada solução de software. A intensidade de participação
varia desde um observador interessado no software até líderes de comunidades. Essa diversidade é
derivada do modelo de produção do software livre, no qual baseou-se o SPB para sua formação
~\cite{alves2009software}. 

A percepção do potencial que representava a participação da sociedade no desenvolvimento do software
e o conceito de bem público foram adaptadas do ponto de vista jurídico, assim levando o Ministério
do Planejamento, Orçamento e Gestão (MP) a formular o conceito de software público. Essa base
jurídico-institucional permitiu a criação de um ambiente virtual (um portal) para a disponibilização
de software como software público. Esse modelo é definido por uma rede que se auto-organiza e cuja
produção se caracteriza pela intensa participação colaborativa entre indivíduos, empresa ou 
prestadores de serviço, universidades e instituições interessadas na evolução de um determinado projeto
de software~\cite{alves2009software}.

O conceito de software público diferencia-se do de software livre em alguns aspectos, destacando-se
a atribuição de bem público ao software. Embora haja algumas diferenças entre o que é um software
livre e um software público brasileiro, há princípios comuns, como a tendência da descentralização na
tomada de decisões, o intenso compartilhamento de informações e os processos de retroalimentação
decorrentes do uso dos artefatos produzidos. Em outras palavras, todo software público é um software
livre também.

Para se disponibilizar um software no Portal do Software Público a equipe do Portal 
oferece um documento chamado Manual do Ofertante onde estão contidas todas as 
informações e procedimentos para disponibilizar um software. neste manual é citada
qual licença de software deve ser aplicada e sobre o registro do software no INPI. 
%TODO atualizar informações

O modelo atual de disponibilização de um software como público é
regido pela Instrução Normativa Nº 01 de 17 de janeiro de 2011. 
Segundo esta IN ao se disponibilizar um software público o mesmo estará se tornando um bem público
que pode ser usado por todos.
%TOOD: verificar a novo IN

A Instrução Normativa 04/2012 indica que os gestores devem consultar as soluções 
existentes no portal do SPB antes de realizar uma contratação de software. Por outro,
a evolução técnica do portal do SPB foi comprometida, desde 2009, ao não acompanhar 
a evolução do seu framework base, o OpenACS. Não houve o lançamento de novas versões do
portal desde então\footnote{http://sbc.org.br/component/flippingbook/book/21}.
%Descreva melhor esse footnote

Em fevereiro de 2016 a equipe da SLTI(Secretaria de Logística e Tecnologia da Informação), ligada ao
MPOG(Ministério do Orçamento Planejamento e Gestão), responsável pelo Portal do Solftware
Público Brasileiro, abriu para consulta pública a minuta para a nova Intrução 
Normativa\footnote{http://www.participa.br/softwarepublico/minuta-da-instrucao-normativa-do-software-publico-software-de-governo-e-projeto-de-software/minuta-da-instrucao-normativa-do-software-publico-brasileiro-e-de-governo/minuta-da-instrucao-normativa}.
Essa nova IN traz novos conceitos e novas regras para a disponibilização de software público sendo elas:
%TODO: tem vir falando da novo IN desde o inicio

\begin{itemize}

\item Qualquer licensa GPL superior a versão 2.0 será aceita.

\item Utilização de modelo de licença livre compatível com a Creative Commons CC-
BY-SA 3.0 BR ou posterior em relação à proteção da marca do software.

\item Foi criado o conceito de Software de Governo, que visa a utilização do Portal
como local para compartilhamento de soluções entre órgãos do governo.

\end{itemize}

As alteraçoes na IN Nº 1 e o lançamento da MInuta só foram possíveis devido 
a reformulação do Portal do Software Público que trasformou o portal em uma 
plataforma colaborativa. Uma nova plataforma para o SPB está foi desenvolvida
pela Universidade de Brasília, através dos seus Laboratórios LAPPIS\footnote{https://fga.unb.br/lappis} 
e MídiaLab\footnote{http://www.midialab.unb.br/} 
em parceria com o Centro de Competência em Software Livre da Universidade de São Paulo(CCSL-USP)\footnote{http://ccsl.ime.usp.br/}.

A nova plataforma do SPB é composta por um conjunto de ferramentas com diferentes funcionalidades
que juntas compõe o novo Portal. Todas as ferramentas são software livre e o que foi desenvolvido pelas equipes da UnB e 
USP foi publicado em repositórios abertos, disponíveis no próprio SPB, as principais são:
\begin{itemize}

\item Colab, ferramenta desenvolvida pela Interlegis\footnote{http://www.interlegis.leg.br/}, órgão ligado 
ao Senado Federal, que têm como objetivo principal a integração com outras ferramentas.

\item Noosfero, é um plataforma de criação de redes sociais que conta com as funcionalidades
que uma rede social provê, bem como blogs, galeria de imagens, entre outras. É mantido pela 
COOLIVRE\footnote{http://colivre.coop.br/} além do SERPRO\footnote{https://www.serpro.gov.br/} e a 
Universidade de Brasília.

\item Gitlab, ferramenta de desenvolvimento colaborativo, projetos no gitlab são mantigos em 
repositório Git que provê: gestão de tarefas (issue tracker), merge requests, 
gestão de marcos (milestones), suporte a integração com plataformas de integração contínua e notificações.

\item GNU Mailman, ferramenta que gerencia listas de e-mail.

\end{itemize}

A arquitetura do novo Portal pode ser vizualizada a seguir\footnote{https://softwarepublico.gov.br/doc/arquitetura.html}.

\begin{figure}[h]
	\centering
	\label{arquitetura}
		\includegraphics[keepaspectratio=true,scale=0.3]{figuras/arquitetura.eps}
	\caption{Arquitetura do novo SPB}
\end{figure}


A nova plataforma do SPB foi lançada para homologação em dezembro de 2014 mas já
está sento utilizada em sua totalidade pelas comunidades do SPB, a antiga versão do Portal 
já foi desativada e suas comunidades foram migradas para a nova plataforma. 
%TODO: atualizar

Disponibilizar um conjunto de ferramentas e melhorar a experiência do usuário no 
ambiente é parte desse processo de reformulação do SPB. Aspectos culturais 
da colaboração em rede para um efetivo uso do que é fornecido na plataforma 
necessitam ser amadurecidos pelo MP junto às comunidades do SPB.

