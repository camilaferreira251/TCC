\begin{anexosenv}

\partanexos

\chapter{Mensagem enviada às comunidades.}
\label{anexo a}

Prezados desenvolvedores do <nome do projeto>,

Sou aluna da Universidde de Brasília, orientanda do Professor
Paulo Meirelles, e meu trabalho de conclusão de curso tem como 
objetivo levantar as barreiras para um desenvolvedor 
começar a contribuir com um projeto de software público. Farei esta pesquisa 
com base na tese de doutorado do Professor Igor Steinmacher, doutorando na USP,
na qual foram levantadas as barreiras para um desenvolvedor começar a contrubuir 
com projetos de software .

Meu resultado final será um pequeno portal onde serão reunidas todas essas informações
a respeito dos projetos pesquisados onde também haverá a oportunidade de cadastro de
novos projetos.

O primeiro passo da minha pesquisa é definir essas barreiras e para tanto é
preciso que me ajudem respondendo o questionário disponibilizado neste link:
<link do questionário>.

Você levará menos de 10 minutos para respondê-lo, e não será necessário que faça
nenhuma identificação.

Desde já agradeço a sua colaboração,

\chapter{Questionário para os desenvolvedores.}
\label{anexo b}

\textbf{Perfil}

Com qual projeto do SPB você contribui?

Há quanto tempo você começou a contribuir com software público?

Esse é o primeiro projeto de software público que você contribui?

Quantos anos de experiencia você tem em desenvolvimento de software?

Qual o seu papel no projeto?

Qual era o seu conhecimento em software público quando você iniciou no projeto?

Você é pago para desenvolver software público?

Essa é sua principal ocupação?

Qual sua idade?

Quanto tempo você gasta contribuindo com software público?

[ ] Menos de 5 horas/semana 

[ ] De 5 a 10 horas/semana

[ ] De 10 a 20 horas/semana

[ ] Mais de 20 horas/semana


\textbf{Motivação}

Qual a sua motivação para contribuir com software público?

Porque especificamente esse projeto?

Como é a receptividade da comunidade?

Porque você continua contribuindo com o projeto?

Entrada no Projeto

Como você começou a contribuir com o projeto?

Como/Porque você começou dessa forma?

Você sabia o que fazer?

Quais foram os passos iniciais?

Quais tarefas você fez quando começou?

Como você escolheu essas tarefas?

Você considera que essas tarefas foram apropriadas para você começar? 

Você desistiu ou mudou a primeira tarefa por algum motivo? Qual?

Você sabe como iniciou a execução tarefa?

Que tipo de informação você precisou para executar a tarefa?

Foi dificil encontrar essa informação?

Atualmente, quais as tarefas que você realiza?


\textbf{Problemas}

Como foi sua primera interação com os membros do projeto? Eles foram receptivos e prestativos?

Qual foi sua frustração com o projeto (considerando o processo de entrada no projeto)?

Como você lidou com isso?

Como você coordenou suas tarefas com os outros membros do projeto?


\textbf{Sugestões}

Quais passos você considera importante para um novato seguir para se torar um contribuidor?

Esses passos foram claramente apresentados a você?

Como você aprendeu esses passos?

Quais mecanismos você propoe para reduzir os problemas enfrentados pelos novatos quando iniciam no projeto?


\chapter{Questionário para os desistentes.}
\label{anexo c}

\textbf{Perfil}

Você tem alguma experiência com projetos desoftware público?

A quanto tempo você contribui com projetos de software público?

Você tem experiencia na liguagens/frameworks usados no projeto?

Você é usuário do software?

Por quanto tempo você contribuiu com o software?

Você foi pago para desenvolver esse solftare público?

Essa é sua principal ocupação?

Qual sua idade?

Quanto tempo você gasta contribuindo com software público?

[ ] Menos de 5 horas/semana 

[ ] De 5 a 10 horas/semana

[ ] De 10 a 20 horas/semana

[ ] Mais de 20 horas/semana

Qual sua motivação para contribuir com software público?


\textbf{Motivação}

Porque você tentou contribuir com o projeto? Qual sua motivação?

Tem interesse em continuar contribuindo?

Quão dificil é achar informação sobre?


\textbf{Processo}

Qual sua primeira ação quando você decide contribuir com um projeto de software público?

Como/Porque você começou desse jeito?

Você estava ciente de quais passos seguir para se tornar um contribuidor?

Você soube como se comportar e quais eram os passos iniciais?

Você já usou os mesmos passos em outro projeto?

\textbf{Problemas}


Qual foi sua frustração com o projeto (considerando o processo de entrada no projeto)?

Como você lidou com isso?

Que tipo de suporte você esperava?

Você sabia como começar?

Você sabia como achar uma tarefa?

Você sabia como coordenar suas tarefas com a comunidade?

Você estava ciente da estrutura do projeto?

Você sabia como contactar pessoas para ajudar você?

Quais mecanismos o projeto tem para te ajudar?

Se você pudesse sugerir mecanismos para ajudar quem começa no projeto o que você sugeriria?


\chapter{Mensagem enviada aos gestores}
\label{anexo d}

Prezado <Nomes dos gestores do projeto>,

Sou aluna da Universidde de Brasília, orientanda do Professor Paulo Meirelles, 
e meu trabalho de conclusão de curso tem como objetivo levantar as barreiras para 
um desenvolvedor começar a contribuir com um projeto de software público. Farei 
esta pesquisa com base na tese de doutorado do Professor Igor Steinmacher, 
doutorando na USP, na qual foram levantadas as barreiras para um desenvolvedor 
começar a contrubuir com projetos de software público.

Como resultado final espero desenvolver um pequeno portal onde serão 
reunidas todas as informações levantadas a respeito dos projetos pesquisados e a 
oportunidade de cadastro de novos projetos.

O primeiro passo da minha pesquisa é definir essas barreiras e para tanto é
preciso que me ajudem respondendo o questionário disponibilizado neste link:
<link do questionário>.

O questionário ficou um pouco grande pois queremos abranger vários assuntos 
portanto caso não queira responder o questionário sozinho podemos marcar uma 
entrevista presencial ou via hangout.

Desde já agradeço a sua colaboração,


\chapter{Questionário para os gestores.}
\label{anexo e}

\textbf{Perfil}

Com qual projeto SPB você contribui?

Há quanto tempo você contribui com projetos de Software Público?

Esse é o primeiro projeto que você contrinbui?

Quantos anos de experiência você tem como desenvolvedor de software?

Há quanto tempo você contribui com esse projeto?

Qual o seu papel atual no projeto?

Quando você começou, qual era o seu conhecimento em software livre?

Você é pago para desenvolver software público?

Essa é sua ocupação principal?

Qual sua idade?

Quanto tempo você gasta contribuindo com software livre?

[ ] Menos de 5 horas/semana 

[ ] De 5 a 10 horas/semana

[ ] De 10 a 20 horas/semana

[ ] Mais de 20 horas/semana

Qual sua motivação para contribuir com software público?

\textbf{Sobre novatos}

O que você acha a respeito de novatos no projeto?

Qual o perfil que você espera de um novato? (exemplificando: conheciemntos técnicos, 
ferramentas que deve conhecer, comprometimento, tipos de issues que deve fechar)

Qual a maneira de identificar um novato que irá continuar contribuindo com o projeto?

Existe alguma atenção especial para novatos no projeto?

\textbf{Processo de entrada na comunidade}

Como os novatos geralmente começam a interagir com a comunidade? (lista de e-mail, 
emails diretos, patches, issue tracker, reportando bugs...)

O que você considera mais apropriado para começar (o que você recomenda)?

Como os novatos ficam sabendo disso?

Quais passos você acha importante para um novato começar a contribuir?

Como os novatos ficam sabendo desses passos?

Você se lembra de algum novato que tenha tido sucesso ao contribuir com o projeto?

Quais as dificuldades que você enfrentou ou sabe que algum novato enfrentou quando 
estava iniciando no projeto?

Você sabe de novatos que desistiram do projeto? Sabe o porquê?

\textbf{Problemas enfrentados}

Na sua opinião: quais são as principais dificuldades enfrentadas pelos novatos ao iniciar 
no projeto? 

Quais são as atividades que os novatos precisam de mais ajuda? 

É facil obter ajuda da comunidade? Os novatos são respondidos em suas primeiras perguntas?

\textbf{Soluções existentes}

Existe algum tipo de ajuda aos problemas para os novatos que estão inicianto o projeto?

Existe algum mecanismo que ajude o novato a iniciar o processo de contribuir com o projeto?

Já pensou em alguma ferramenta para dar suporte aos usuários novatos?


\end{anexosenv}

