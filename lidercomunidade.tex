Usuário líder de comunidade

O novo ambiente do Portal do Software Público disponibilizará um ambiente prático
que será composto por uma repositório, lista de e-mail e wiki para documentação.

No primeiro acesso neste ambiente prático o líder da comunidade deverá criar seu
login fornecendo nome, sobrenome, e-mail e nome de usuário. Após este pequeno 
cadastro já será possível fazer a autenticação no ambiente, no primeiro acesso
será pedida a confirmação do e-mail cadastrado onde o usuário colocará sua senha 
de acesso e feito isso terá sua autenticação autorizada. Este procedimento ocorrerá 
somente no primeiro acesso, após este a autenticação ocorrerá normalmente.

Todos os ambientes práticos deverão conter uma lista de discussão que deverá ser criada
no primeiro acesso do usuário líder da comunidade prática. Para a criação desta lista
[TODO: definir os passos para a criação de grupos no  colab]

A nova comunidade deverá também ter um repositório que não necessariamente precisa
ser de cógigo-fonte, podendo ser, por exemplo, de documentos. Logo após a criação
do repositório o usuário lider da comunidade poderá dar o commit inicial de conteúdo 
para o repositório. A criação do repositório proporcionará a criação de uma wiki que é
de existência origatório na comunidade e é facilmente habilitada na seção 'Setings' 
do repositório.

Será disponibilizada uma Issue/Tracker para o repositório que é habilitada também 
na seção 'Setings' do repositório. Issue/Tracker é fundamental para a gerência das 
atividades pendentes do repositório e pode ser de 3 tipos: bug, nova funcionalidade 
ou documentação,  e podem ainda ser assinadas para um membro do repositório. O líder 
da comunidade poderá também criar uma milestone, que nada mais é que um período de 
tempo para execução de determinadas atividades e vincular uma Issue a uma milestone 
facilitando ainda mais a gerência das atividades dentro do repositório.

Somente o usuário líder da comunidade poderá dar permissão de alteração no repositório
para novos membros, para tanto o novo membro da deverá estar cadastrado no ambiente
e o líder da comunidade adicionará o novo membro acessando o menu 'Setings' do
repositório seguido do menu 'Menbers', nesta opção aparecerão todos os usuário do
repositório e clicando na opção 'New project membre' basta colocar o nome do novo
membro e escolher o tipo de permissão que será dada a este novo membroe clicar em 
'Add users' e o novo membro já estará com permissões.

