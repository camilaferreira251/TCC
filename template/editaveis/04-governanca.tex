\chapter[Governança]{Governança}

\section{Teoria Geral da Administração}

A preocupação em racionalizar, padronizar e prescrever normas é estudada desde a 
Administração Científica proposta por Taylor. Segundo essa proposta a gerência 
deve seguir 4 princípios:\cite{chiavenato2001teoria}

\begin{enumerate}
\item Princípio de planejamento. Substituir no trabalho o critério individual 
do operário, a improvisação e a atuação empírico-prática, por métodos baseados 
em procedimentos científicos. Substituir a improvisação pela ciência através
do planejamento do método de trabalho.
 
\item Princípio de preparo. Selecionar cientificamente os trabalhadores de acordo 
com suas aptidões e prepará-los e treiná-los para produzirem mais e melhor, de 
acordo com o método planejado. Preparar máquinas e equipamentos em um arranjo 
físico e disposição racional.

\item Princípio do controle. Controlar o trabalho para se certificar de que está 
sendo executado de acordo com os métodos estabelecidos e segundo o plano previsto. 
A gerência deve cooperar com os trabalhadores para que a execução seja a melhor 
possível.

\item Princípio da execução. Distribuir atribuições e responsabilidades para 
que a execução do trabalho seja disciplinada.
\end{enumerate}

A Adminitração científica deu pouca importância para os aspectos humanos de uma
organização, restringindo-se apenas a fatores relacionados diretamente com o 
cargo ou função do operário, embora saibamos que uma organização é constituída 
por pessoas.

Paralelamente ao surgimento da Administração Científica de Taylor surgia a 
Administração Clássica, que diferentemente da teoria de Taylor que se caracterizava
pela ênfase na tarefa realizada pelo operário, se caracterizava pela ênfase na 
estrutura da organização com o mesmo objetivo de buscar a eficiência das
ornanizações.

Na Teoria Clássica, preconizada por Fayol, ao contrário, partia-se do todo 
organizacional e da sua estrutura para garantir eficiência a todas as partes 
envolvidas, fossem elas órgãos (como seções, departamentos etc.) ou pessoas 
(como ocupantes de cargos e executores de tarefas).
%
Fayol define o ato de administrar como: prever, organizar, comandar, coordenar 
e controlar. As funções administrativas envolvem os elementos da Administração, 
isto é, as funções do administrador:\cite{chiavenato2001teoria}

\begin{enumerate}

\item Prever. Visualizar o futuro e traçar o programa de ação.

\item Organizar. Constituir o duplo organismo material e social da empresa.

\item Comandar. Dirigir e orientar o pessoal.

\item Coordenar. Ligar, unir, harmonizar todos os atos e esforços coletivos.

\item Controlar. Verificar que tudo ocorra de acordo com as regras estabelecidas 
e as ordens dadas.

\end{enumerate}

Com o surgimento da Abordagem humanística e a Teoria das Relações Humanas,
a teoria Administrativa passa por uma revolução conceitual: a transferência da 
ênfase antes colocada na tarefa (pela Administração Científica) e na estrutura 
organizacional (pela Teoria Clássica) para a ênfase nas pessoas que trabalham 
ou que participam nas organizações. A Abordagem Humanística faz com que a
preocupação com a máquina e com o método de trabalho e a preocupação com a 
organização formal e os princípios de Administração cedam prioridade para a 
preocupação com as pessoas e os grupos sociais.


\section{Conceito de Governança}

Governança pode ser definida como o sistema pelo qual as organizações 
são monitoradas envolvendo o relacionamento entre todos os integrantes,
são recomendações objetivas alinhando interesses na intenção de prezervar e 
otimizar o valor da organização, em outras palavras é um conjunto de processos
que regulam a maneira como uma organização é dirigida e estuda as relações entre 
os diferentes atores que a compõe

O conceito de governança não é novo e, segundo relatório da ONU, significa o processo 
de tomar decisões e o processo de escolher as decisões que serão implementadas.
%
Governança pode ser usada em diferentes contextos, seja corporativo,de governo ou 
local.

Mais especificamente para tecnologia da informação um sistema de governança
estabelece mecanismos, estruturas e incentivos, que compõem o sistema de controle de 
gestão da empresa e direciona o comportamento dos administradores para o cumprimento 
dos objetivos estipulados pelos acionistas/proprietários (MARTIN, 2004).
%
Além de "Capacidade organizacional de controlar a formulação e implementação da 
estratégia de TI e guiar a mesma na direção adequada com o propósito de gerar 
vantagens competitivas para acorporação” (THE MINISTRY OF INTERNATIONAL TRADE 
AND INDUSTRY, 2003).

Os conjuntos de processos definidos pela governança visam além de organizar, 
dar transparência ao trabalho a ser realizado.

Processo é um conjunto sequencial de ações baseado em estratégia para alcançar
um objetvo. Para definir uma estratégia são necessarios 3 passos, identificar a
situação do ambiente, identificar a situação da organização e por fim é comparada
a situação do ambiente e da organização e identificados os cursos alternativos
de ação e escolhida a melhor alternativa.  

Trazendo para a realidade do Novo Portal do Software público, a governaça pode ser usada para 
definir um processo que auxilie na gestão do Portal como um todo, e mais especificamente,
das comunidades que nele estão contidas, imprimindo organização e controle para atender 
a demanda não somente do governo mas também dos usuários da nova plataforma.

Para que este desenho faça sentido é necessário que seja colcado no processo somente o que 
realmente tem valor de negócio para o usuário e para os gestores.

A definição da governaça dará subsídios para sejam notados os padrões de uso e colaboração na
nova plaforma, podendo assim ser geradas algumas 'boas pŕaticas' para uso e colaboração
visando a manutenebilidade da plataforma.

\section{Modelo SISP}
	






