Capítulo Sobre Governança

Conceito de Governaça

Governança pode ser definida como o sistema pelo qual as organizações 
são monitoradas envolvendo o relacionamento entre todos os integrantes,
são recomendações objetivas alinhando interesses na intenção de prezervar e 
otimizar o valor da organização, em outras palavras é um conjunto de processos
que regulam a maneira como uma organização é dirigida e estuda as relações entre 
os diferentes atores que a compõe

O conceito de governança não é novo e, segundo relatório da ONU, significa o processo 
de tomar decisões e o processo de escolher as decisões que serão implementadas.
%
Governança pode ser usada em diferentes contextos, seja corporativo,de governo ou 
local.

Mais especificamente para tecnologia da informação um sistema de governança
estabelece mecanismos, estruturas e incentivos, que compõem o sistema de controle de 
gestão da empresa e direciona o comportamento dos administradores para o cumprimento 
dos objetivos estipulados pelos acionistas/proprietários (MARTIN, 2004).
%
Além de "Capacidade organizacional de controlar a formulação e implementação da 
estratégia de TI eguiar a mesma na direção adequada com o propósito de gerar 
vantagens competitivas para acorporação” (THE MINISTRY OF INTERNATIONAL TRADE 
AND INDUSTRY, 2003).

Os conjuntos de processos definidos pela governança visam além de organizar, 
dar transparência ao trabalho a ser realizado.

Processo é um conjunto sequencial de ações baseado em estratégia para alcançar
um objetvo. Para definir uma estratégia são necessarios 3 passos, identificar a
situação do ambiente, identificar a situação da organização e por fim é comparada
a situação do ambiente e da organização e identificados os cursos alternativos
de ação e escolhida a melhor alternativa.  

Trazendo para a realidade do Novo Portal do Software público, a governaça pode ser usada para 
definir um processo que auxilie na gestão do Portal como um todo, e mais especificamente,
das comunidades que nele estão contidas, imprimindo organização e controle para atender 
a demanda não somente do governo mas também dos usuários da nova plataforma.

Para que este desenho faça sentido é necessário que seja colcado no processo somente o que 
realmente tem valor de negócio para o usuário e para os gestores.

A definição da governaça dará subsídios para sejam notados os padrões de uso e colaboração na
nova plaforma, podendo assim ser geradas algumas 'boas pŕaticas' para uso e colaboração
visando a manutenebilidade da plataforma.


	






